% !TEX TS-program = pdflatex
% !TEX encoding = UTF-8 Unicode
% These two above lines are to specify the encoding scheme of THIS, the input file.

% DOCUMENT HEADER.

% Declare the document.
\documentclass[a4paper, oneside, 11pt, titlepage, onecolumn, openright]{report}

% INCLUDE REQUIRED PACKAGES.

% Package to support disabling output for fast syntax checking.
\usepackage{syntonly}

% Package to support logical operations.
\usepackage{ifthen}	

% Package adding some useful symbol macros.
\usepackage{textcomp}

% Package which allows the use of graphs, and ensures they work with pdftex.
\usepackage[pdftex]{color, graphicx}

% Package which sets input encoding.
\usepackage{inputenc} 

% Package allowing improved page layout configuration.
\usepackage{geometry}

% Package which allows better configuration of section heading appearance.
\usepackage{sectsty}

% Package which adds support for fancy headers.
\usepackage{fancyhdr}

% Package which modifies paragraph spacing options.
\usepackage[parfill]{parskip}

% Packages which adds support for better looking tables.
\usepackage{booktabs}

% Package which adds support for better arrays/matrices in maths environments.
\usepackage{array}

% Package which adds support for flexible list environments.
\usepackage{paralist}

% Package which adds support for improved verbatim blocks.
\usepackage{verbatim}
\usepackage{moreverb}

% Package which allows adding multiple figures to a single float.
\usepackage{subfig}

% Package for binding the bibligraphy into the ToC.
\usepackage[nottoc,notlof,notlot]{tocbibind}

% Package allowing customization of the ToC, LoF and LoT.
\usepackage[titles,subfigure]{tocloft}

% Package which allows importing of multi-page PDF documents.
\usepackage{pdfpages}

% Package which lets you put in source code
\usepackage{listings}

% OUTPUT CONFIGURATION OPTIONS.

% Disable generation of output if we just want to check the syntax.
%\syntaxonly

% If this document is going to be printed rather than viewed online, we make some optimizations.
\newcommand{\pdfforprinting}

% DOCUMENT PREAMBLE.

% Specify the input file encoding.
\inputencoding{utf8}

% Configure page layout options.
\geometry{a4paper}
\geometry{margin=2.5cm}
\geometry{portrait}

% Update PDF page layout to match page geometry.
\pdfpagewidth=\paperwidth
\pdfpageheight=\paperheight

% Set the title attributes.
\newcommand{\projecttitle}{ValleyForge Embedded Toolchain}
\newcommand{\documenttitle}{Maintenance Guide}
\title{\projecttitle \\ \documenttitle}
\author{Zac Frank \& Paul Bowler}
\date{\today}

% Modify the paragraph style.  This adds roughly a single line between paragraphs.
\setlength{\parindent}{0pt}
\setlength{\parskip}{1em plus 0.5ex minus 0.2ex}

% Modify the line spacing.  This allows a line spacing factor for proofing copies.
\linespread{1.0}

% Select fonts to use for sections.
\allsectionsfont{\sffamily\mdseries\upshape}

% Configure how headers behave.
\pagestyle{fancy} % Options are: empty , plain or fancy.
\renewcommand{\headrulewidth}{0pt}
\lhead{}\chead{}\rhead{}
\lfoot{}\cfoot{\thepage}\rfoot{}

% Configure some improved support for including source code.
\def\sourcetabsize{4}
\newenvironment{sourcestyle}{\begin{scriptsize}}{\end{scriptsize}}
\def\sourceinput#1{\par\begin{sourcestyle}\verbatimtabinput[\sourcetabsize]{#1}\end{sourcestyle}\par}

% Configure the hyperref package.  Due to some magic, this MUST be last in the preamble.
\usepackage[pdftex, bookmarks, pdfstartview={XYZ null null 1.25}]{hyperref}
\ifthenelse{\not{\isundefined{\pdfforprinting}}}
	{
		% Remove colored links since the PDF is going to be printed.
		\hypersetup{colorlinks=false}}  
	{
		% Add coloured links since the PDF will be viewed online.
		\hypersetup{colorlinks=true}
		\hypersetup{linkcolor={blue}}
	}
\hypersetup{pdftitle=\projecttitle~--~\documenttitle}
\hypersetup{pdfauthor=\@author}

% BEGIN THE DOCUMENT PROPER.
\begin{document}

	% Create the title page.
	\maketitle
	
	% Insert the table of contents (ToC).
	\clearpage
	\tableofcontents

	
	% THE ACTUAL GIST OF THE DOCUMENT STARTS HERE.

	% An article has sections as the top level sectioning command.	

	% Make sure the body starts on a new page.
	\clearpage

\chapter{Linux Environment}
		\label{C:Environment}

		This section details the installation of the toolchain.
		\section{Installation }
			\label{s:Installation}
			The following are assumed:
			
\begin{itemize}
			\label{Requirements}
\item The user has a firm grasp on C++ programming.

\item The user has a rudimentary understanding of Bash scripting. (A good tutorial can be found at \url{http://linuxcommand.org/writing\_shell\_scripts.php})

\item The user knows how to use Git to update the remote repository (a tutorial can be found at \url{http://stackoverflow.com/questions/315911/git-for-beginners-the-definitive-practical-guide})

\item The user has the latest version of Ubuntu installed. If not, the latest version can be downloaded from \url{http://ucmirror.canterbury.ac.nz/linux/ubuntu-releases/}. This image can be burned onto a disc, the computer booted from the disc, and on-screen instructions followed.

\end{itemize}
			Once Ubuntu has been installed, we need to install the packages that do not come preinstalled with Ubuntu but are necessary for the toolchain to run. Most of these packages can be installed using the Debian Package repository. To install these packages, open the terminal, and enter:
			
\begin{lstlisting}[frame=trBL, breaklines=true, language = bash]
sudo apt-get install byacc flex gcc-avr avr-libc
\end{lstlisting}


Next the toolchain will be installed.. Navigate to the directory where the toolchain will reside. 
In the command line, type \newline

\begin{lstlisting}[frame=trBL, breaklines=true, language = bash]
git clone (PUT REPOSITORY URL HERE)
\end{lstlisting}
to download the toolchain to your working directory.

See the User Guide for instructions on how to use the toolchain.
\pagebreak
\chapter{Directory Structure}
			\label{C:Directory Structure}
			
The ValleyForge Folder contains eight subdirectories, each of which will be briefly explained, including each one's own subdirectories. More complex aspects are described in more detail in later sections.

\section{bin}
			\label{s:bin}
			This directory contains the binaries that are produced when a component is compiled. Binaries are copied here from where they are compiled in the \texttt{tmp} directory, and are named after the component. If an accompanying bootloader is also being built, this is also placed in this directory with the name \texttt{COMPONENTNAME\_bootloader.hex}. The toolchain currently has no mechanism with which to delete binaries from this folder so at this stage they must be deleted manually (although when a component gets rebuilt the old binary is properly replaced by the newer automatically).
			
\section{bld}
			\label{s:bld}
			The bld subdirectory contains (almost) all the bash scripts which implement the toolchain functionality. There are three scripts which provide the main interface to the user. These call other scripts in other subdirectories and are the only scripts necessary to be called directly by the user.
			
\begin{itemize}
			\label{bldscripts}
\item \label{VFstartsummary}\textbf{\texttt{VFstart}}. This script should be the first called. It is used to create new components, edit, delete, create new files, change user info, e.t.c. A detailed description of this script can be found in section \S\ref{s:VFstart}.

\item \label{buildsummary}\textbf{\texttt{build}}. This script is used to build components and bootloaders. When run without any arguments, it builds the component currently set as active.  A detailed description can be found in section \S\ref{s:build}.

\item \label{resettcsummary}\textbf{\texttt{resettc}}. This script resets the toolchain to its default values, i.e erases user info, default component. Does not erase the source directory.

\end{itemize}

Two other files are contained in this directory: 

\begin{itemize}

\item \textbf{\texttt{build\_configs.cfg}}. This file is used by many other scripts. Usually in the form of 
\begin{lstlisting}[frame=trBL, breaklines=true, language = bash]
source bld/build_configs.cfg
\end{lstlisting}[frame=trBL, breaklines=true, language = bash]
which runs the file as a script. It contains bash function defintions, each of which does nothing except assign various fields. Therefore when the user calls the function, a collection of variables become assigned, corresponding to the build configuration selected. Build configurations correspond to the different combinations of platform and microprocessor, and each has its specific variables to be set (such as Makefile used, flash address to start from).

\item \textbf{\texttt{default\_user\_config.cfg}}. A file which the \texttt{user\_config.cfg} file gets restored to when the \texttt{resettc} script is called.

\end{itemize}


\subsection{bld/common}
			\label{ss:bld/common}
			Located here are small scripts that are used by many other scripts throught the \texttt{bld} subdirectories. 
			
\begin{itemize}

\item \textbf{\texttt{clear\_bconf\_vars}}. A safety mechanism. Clears build configuration variables so that the wrong values are not used by accident. This is important because tests are often performed throughout scripts to see whether a certain variable is defined. If a variable remains defined, but with the wrong value, this can lead to serious errors.

\item \textbf{\texttt{def\_colours}}. Each script, when displaying text to the user, uses a colour scheme for different types of messages, defined in this file. Colours are explained in more detail in section \S\ref{s:Colours}.

\item \textbf{\texttt{load\_build\_configs}}. This checks the \texttt{build\_configs.cfg} file to see if there are any valid build configurations, and loads it (sets the appropriate variables). See how the build configurations work in section \S\ref{s:buildConfigs}.

\end{itemize}			
			
\subsection{bld/create}
			\label{ss:bld/create}
			In this directory there are four scripts which are called when a new instance of something is created.
			
\begin{itemize}

\item \textbf{\texttt{create\_component}}. Creates a new component. Brings up an interface (if not in batch mode) prompting the user to enter component details, then creates a new subdirectory in src, and asks to create source files. A detailed description can be found in section \S\ref{s:createComponent}.

\item \textbf{\texttt{create\_files}}. Creates new source files by copying and editing templates, and places them in the specified source directory. A detailed description can be found in section \S\ref{s:createFiles}.

\item \textbf{\texttt{create\_library}}. Similar to creating a new component, creates a new library. Also asks to create source files after setting up the library.

\item \textbf{\texttt{create\_library\_files}}. Creates new library files from templates and places them in the library source path.

\end{itemize}			

\subsection{bld/delete}
			\label{ss:bld/delete}
			These scripts are used for deleting components and libraries.
\begin{itemize}

\item \textbf{\texttt{delete\_component}}. Brings up a list of the available components. Deletes the source files and folder for the specified component. If the component is the active one, resets the active component.

\item \textbf{\texttt{delete\_library}}. Brings up a list of the available libraries. Deletes the specified library from the toolchain.

\end{itemize}

\subsection{bld/edit}
			\label{ss:bld/edit}
\begin{itemize}

\item \textbf{\texttt{edit\_user\_config}}. Changes toolchain variables, such as user name and subsystem.

\end{itemize}

\subsection{bld/other}
			\label{ss:bld/other}
\begin{itemize}

\item \textbf{\texttt{first\_run}}. This script is called when the toolchain is first run, and prompts the user to enter their information.

\end{itemize}

\subsection{bld/preprocess}
			\label{ss:bld/preprocess}
\begin{itemize}

\item \textbf{\texttt{preprocess}}. This script is used to change the contents of source files. It edits template files before moving them to the source directory, and it inserts variables into source files before compiling at build time. A more detailed description can be found in section \S\ref{s:Preprocess}.

\end{itemize}

\section{doc}
			\label{s:doc}
			This is where the User Guide and this manual are stored, as well as the latex files for both. To edit these files, you need to have a \LaTeX\ editor (you can acquire Texmaker through the Ubuntu Software Center). Please keep both files up-to-date as necessary.
			
\section{lib}
			\label{s:lib}
			User defined libraries are stored here. They can be accessed using \texttt{\#include lib/LIBRARY\_NAME} in component source files. For development variations of toolchain, the contents of this folder are not under version control.
			
\section{res}
			\label{s:res}
			This folder stores all the toolchain HAL libraries, bootloaders, makefiles, FreeRTOS files, and basically any external resources needed to build programs. The various sub-directories:
			
\subsection{res/arm}
			\label{ss:res/arm}
			This contains a makefile and other files specific to the OMAP4430. I'm really not too sure what's going on here, so hopefully Paul will have filled this in before I leave here. A detailed description of what's going on can be found in section [].
			
\subsection{res/avr}
			\label{ss:res/avr}
			This contains all files specific to the AVR architecture. Mostly HAL and FreeRTOS files are contained in this subdirectory. The files at the top level:
			
\begin{itemize}

\item \textbf{\texttt{bconf\_specific\_config}}. This is a bash script called when a new component is created. Some options are specific to certain build configurations, and therefore need not be set with each component. 

\item \textbf{\texttt{bload\_specific\_config}}. This is a script called when a bootloader is added to a component. Some options are specific to certain bootloaders, and do not need to be set with each bootloader. For instance, the AVR bootloader configuration needs to know which EEPROM address to store the shutdown state flag in, whereas the AVR32 microprocessor has no EEPROM to store such a flag.

\item \textbf{\texttt{Makefile}}. This makefile is used to compile and link all avr components. It is copied to the tmp directory at build time, preprocessed to edit various fields to those needed for the built component, and then run to compile the component.

\end{itemize}					

There are three folders:

\begin{itemize}

\item \textbf{\texttt{bootloader}}. This contains source files for the avr bootloader, including a generic file and respective modules. At time of writing it has not yet been implemented to communicate with a programmer, nor does it write to flash memory. It currently flashes an LED (D3) and then runs the application when a button (B3) is pressed.

\item \textbf{\texttt{freertos}}. FreeRTOS consists mainly of common files, plus a few port-specific files. The port specific files for FreeRTOS on the various AVR microprocessors are all contained in this subdirectory.

\item \textbf{\texttt{hal}}. The AVR specific implementations of the ValleyForge HAL are contained here in \texttt{.cpp} files. At time of writing the following modules have been implemented:
\begin{itemize}
\item gpio
\item tc
\item watchdog
\item mem
\end{itemize}

Creating new HAL is detailed in section \S\ref{C:HAL}.

\end{itemize}

\subsection{res/avr32}
			\label{ss:res/avr32}
			
Contained here are the port specific files for the UC3 microprocessors (so far just the AT32UC3C0512C).
While the avr compiler is installed directly onto the computer and can be accessed via the terminal, the AVR32 compiler is located within the toolchain, and is called from there. 

There will be one directory initally in the toolchain, but the avr32comp.tar file gets expanded into another whenever the AVR32 compiler is used.

The folder:

\begin{itemize}

\item \textbf{\texttt{freertos}}. This contains the AVR32 port specific FreeRTOS files, which are transferred to the compile directory at build time.

\end{itemize}

The two files:

\begin{itemize}

\item \textbf{\texttt{avr32comp.tar}}. This is an uncompressed tarball, it contains the AVR32 utilities used for building components for AVR32 microprocessors.

\item \textbf{\texttt{Makefile}}. This is the makefile used whenever a component is compiled for AVR32. Like the AVR makefile, it gets preprocessed at build time, specific to the build configuration and files being compiled.

\end{itemize}

\subsection{res/common}
			\label{ss:res/commmon}
			This contains files common to all architectures. It could probably contain FreeRTOS, although currently this is in its own subdirectory in \texttt{res}. Currently it contains the header files for the HAL. These get copied to the build directory at compile time, along with the architecture specific HAL files from the respective directories.
			
		
\subsection{res/freertos}
			\label{ss:res/freertos}
			Here are contained the common files for FreeRTOS which are copied to the build directory at compile time. As well as this, there is the folder FreeRTOSV7.1.0 which contains all the latest files from FreeRTOS, including all current official ports. 
			
\subsection{res/templates}
			\label{ss:res/templates}
			All the template files which are used to outline the structure of a new component are stored here. They are fairly self-explanatory. The freertos template contains a simple task to build on. FreeRTOS is explained more clearly in section \S\ref{C:FreeRTOS}.
			
\subsection{res/vendor}
			\label{ss:res/vendor}
			This is where external libraries should be stored. It's not really up and running properly at this stage, but the idea is that eventually we can update external libraries with Git Submodules into this folder, then put our own versions of these libraries into \texttt{lib}, and git merge them whenever there are updates. At the moment this contains the Atmel Software Framework, which is a massive collection of libraries and utilities for both AVR and AVR32. For developing new functionality, it is often useful to look in there first.
			
\section{src}
			\label{s:src}
			Here is where the user's source code is placed, as well as other component specific files, such as the component configuration file, and the \texttt{FreeRTOSconfig.h} header file, which specifies FreeRTOS parameters, should the user be implementing FreeRTOS as part of their project.
			Currently, the code is all placed into one folder named after the component, as a subdirectory of \texttt{src}. However, future functionality could be added to support subdirectories of the component folders, in case the user wanted to include sub-modules in different folders for clarity.
			At compile time, the source files in the \texttt{src/component} folder get copied to the \texttt{tmp} directory and preprocessed to have all the correct information.
			
\section{tmp}
			\label{s:tmp}
			The \texttt{tmp} directory is used as a place to compile and link the components and bootloaders. At the start of every build session, it is cleared of all files and folders. Files get copied in from various sources, and preprocessed, then compiled and linked. The resulting \texttt{hex} file is copied into the \texttt{bin} directory. 
			The directory iscleared after building, but the files may be retained by adding the flag ``\texttt{-r}'' to the build script command (or ``\texttt{--retain}'').
			
\section{var}
			\label{s:var}
			Currently all this contains is the user info in the form of \texttt{config.cfg}. It has a few fields which are user-specific. May be added to. Not really much else to say about it.
			
\section{TODO}
			\label{s:TODO}
			This is updated with everything that needs to be added or changed with the toolchain. When something is fixed, remove it from the \texttt{TODO} file, and mention it in the comment of your Git commit.
\pagebreak			
\chapter{Standard Bash Structures}
			\label{C:BashStructures}
			Although an understanding of bash scripting is assumed, there are many ways to acheive certain functionalities in bash. However, we want a standard for how certain aspects are carried, to make the scripts easier to read and more easily maintainable by multiple engineers.
			
\section{Script location}
			\label{s:ScriptLocation}
			At the start of every script we need to figure out what the absolute path to the Toolchain directory is. This is the one from the create\_files script:
			
\begin{lstlisting}[frame=trBL, breaklines=true, language = bash]
# Determine what the absolute path to the root of the toolchain is.
SCRIPT=`readlink -f $0`
SCRIPTPATH=`dirname $SCRIPT`
TCPATH=$(echo $SCRIPTPATH | sed 's/\/bld.*//')
\end{lstlisting}
After this, the variable \$\{TCPATH\} can be used anywhere in the script to point to a specific file or folder within the toolchain. The toolchain standard is to use this absolute path to files and folders, rather than a relative path (such as ../../bin/myfile.hex).
			
\section{Colours}
			\label{s:Colours}
			The ValleyForge Toolchain regularly prints messages to the terminal. In order to make these messages clear and their type discernable, different types of messages are given different colours. All scripts, after determining the location of the toolchain, have the code:

\begin{lstlisting}[frame=trBL, breaklines=true, language = bash]
source $TCPATH/bld/common/def_colours

\end{lstlisting}			
			
which loads the constants defined in the colours script. These colours are then used as such:

\begin{lstlisting}[frame=trBL, breaklines=true, language = bash]
echo -e -n "${GREEN}Enter basenames for the files to be created: (Space separated) ${NO_COLOUR}"
\end{lstlisting}					
			
			Note that the \$\{NO\_COLOUR\} tag is always used at the end of a sentence, to clear the applied colour. This is important, as any messages that do not come from the toolchain will all be in the default terminal text colour.
			
			The colours used are
			
\begin{itemize}

\item \textbf{\texttt{Green}}. This is used for prompts, when the user is required to input a parameter or answer a question.

\item \textbf{\texttt{Cyan}}. Toolchain status messages use this colour, to inform the user what is happening in the background.

\item \textbf{\texttt{Yellow}}. Warning messages use yellow to warn that something is not optimal, but is not unexpected so is not considered an error.

\item \textbf{\texttt{Red}}. Red is for error messages.

\end{itemize}

			For vital information, use the colour tags with BOLD at the front, such as \texttt{BOLD\_RED} for fatal error messages.

\section{Script Parameters}
			\label{s:ScriptParameters}
			
			When a script is called, it often has some parameters passed to it to determine its course of action. To demonstrate the method of parsing these parameters and showing the user how to manipulate them, an example using the \texttt{create\_files} script is provided.
\begin{lstlisting}[frame=trBL, breaklines=true, language = bash]
PROGNAME=${0##*/} 
SHORTOPTS="hbn:c:t:"
LONGOPTS="help,batch,name:,component:,type:"

# Use 'getopt' to parse the command line options.
ARGS=$(getopt -s bash --options $SHORTOPTS --longoptions $LONGOPTS --name $PROGNAME -- "$@")
eval set -- "$ARGS"

# Handle the parsed parameters.
while true; do
	# Select the appropriate behaviour for each parameter.
	case $1 in
		-h|--help)
			# Just print the usage message and then exit.
			usage
			exit 0
			;;
		-b|--batch)
			# Select batch mode.
			BATCHMODE=1
			;;
		-n|--name)
			# Specify the name of the files to create.
			shift
			FILENAMES="$FILENAMES$1 " # NOTE - The space is intentional!
			;;
		-c|--component)
			# Specify the name of the component to add files to.
			shift
			COMPONENT="$1"
			;;
		-t|--type)
			# Specify the type of files to create.
			shift
			TYPE="$1"
			;;
		--)
			# We're done parsing options.  Anything else must be parameters.
			shift
			FILENAMES="$FILENAMES$* " # NOTE - The space in intentional!
			break
			;;
		*)
			# Anything else must be parameters.
			shift
			FILENAMES="$FILENAMES$* " # NOTE - The space in intentional!
			break
			;;
	esac

	# Advance on to the next parameter.
	shift
done
\end{lstlisting}

To add a new parameter, for example ``size'', we would first add into long and short opts:

\begin{lstlisting}[frame=trBL, breaklines=true, language = bash]
SHORTOPTS="hbn:c:t:s:"
LONGOPTS="help,batch,name:,component:,type:,size:"
\end{lstlisting}

The ``:'' character indicates that it expects some value after the option is called. If it was a flag which didn't need a parameter, it would not need this ``:''.

Next, add it to parameter handling section:
\begin{lstlisting}[frame=trBL, breaklines=true, language=bash]
-t|--type)
			# Specify the type of files to create.
			shift
			TYPE="$1"
			;;
-s|--size)
			# Specify the size of files to create
			shift
			SIZE="$1"
			;;
\end{lstlisting}

and add it to the usage menu in the ``usage'' function:

\begin{lstlisting}[frame=trBL, breaklines=true, language = bash]
Options:
	-h --help			Show this message.
	-b --batch			Operate in 'batch mode', wherein there are no interactive prompts.
	-n --name <File Name>		Specify the name of the files to add to the current project.
	-c --component			Specify the name of the component to add the files to.  Defaults to the 'active component'.
	-t --type <File Type>		Specify the kind of files to create (either 'C' or 'CPP').
	-p --platform			Specify the platform for which the files will be created (BareMetal/freertos)
	-s --size				Specify the size of the files to create.
EOF
\end{lstlisting}

You can now use the variable \$\{SIZE\} in the rest of the script. However, it would be advised to test if the value provided is valid and/or non-empty.

\section{User Prompts}
			\label{s:UserPrompts}
			Often a script will ask for user input before it can continue. There are four main types of prompts:
			
\begin{itemize}

\item \textbf{\texttt{Given Words, First Letter Underlined}}. This is used in the VFstart script near the beginning, where preset and unchanging options are given. The user hits the key corresponding to the underlined letter in each option. 

\item \textbf{\texttt{Enumerated List}}. A list of options is given with a number before each option. The user enters a number and presses the Enter key. This is useful because more parameters can be added, or entirely new ones added through an external variable, very easily and elegantly.

\item \textbf{\texttt{Text Input}}. Here the user enters some text, followed by the Enter key, which is then used by the toolchain. An example where this is used is when the user is asked to enter the name for their new component.

\item \textbf{\texttt{Yes/No Prompts}}. ``Y'' or ``N'' key is pressed (Either capital or lower-case).
\end{itemize}			
			
\subsection{Given Words}
			\label{ss:GivenWords}
			An example from the script VFstart:
			
\begin{lstlisting}[frame=trBL, breaklines=true, language = bash]
while :
	do
	# Print the submenu prompt to the user.
	echo -e "${GREEN}Please select an option:\n${NO_COLOUR}"
	echo -en "\tCreate ${UNDERLINED}C${NO_COLOUR}OMPONENT    Create ${UNDERLINED}F${NO_COLOUR}ILES    Create ${UNDERLINED}L${NO_COLOUR}IBRARY    ${UNDERLINED}B${NO_COLOUR}ACK"

	# Read a single character input from the user and select the appropriate response.
	read -s -n 1
	echo -e "\n" # NOTE - This is required since read won't add a newline after reading a single character.
	case "$REPLY" in
		"C" | "c" )
			# Run the create component script.
			bash $TCPATH/bld/create/create_component
			;;

		"F" | "f" )
			# Run the create files script.
			bash $TCPATH/bld/create/create_files
			;;

		"L" | "l" )
			# Run the create library script.
			bash $TCPATH/bld/create/create_library
			;;

		"B" | "b" | "Q" | "q" | "X" | "x" )
			# Go back on level.  Force redrawing the top level menu, even if we aren't in block mode.
			REDRAW_TOP=1
			break
			;;

		*)
			# Any other option is invalid.  We print a message to that effect and try again.
			echo -e "${RED}Invalid choice. Try again or press Q to quit.\n${NO_COLOUR}"
			continue
			;;
	esac
				
	# We're done here.
	break
done		        
\end{lstlisting}

				The options for the ``\texttt{read}'' function here dictate that it should only wait for one character from the user before it parses it.

\subsection{Enumerated List}
			\label{ss:EnumeratedList}	
				This is a compact method and is shown below in the create\_component script for choosing a platform:
				
\begin{lstlisting}[frame=trBL, breaklines=true, language = bash]				
# Create a menu of choices and have the user select one.
select PLATFORM in $PLATFORMS
	do
		# Check if the selected platform is actually valid.
		PLATFORM=$(echo "$PLATFORMS" | grep -w -o "$PLATFORM")
		if [ -z "$PLATFORM" ]; then
			# The selected platform was not in the list of platforms, so the user is apparently a moron.
			echo -e "${RED}Invalid choice.  Try again.\n${NO_COLOUR}"
		else
			# A legitimate option was selected, so we can go now.
			echo -e "${CYAN}Selected platform $PLATFORM.\n${NO_COLOUR}"
			break
		fi
	done				
\end{lstlisting}			


In this case, the variable \$\{PLATFORMS\} is a set of words separated by spaces. The list of words is enumerated, and the option selected is assigned to the variable \$\{PLATFORM\}. 

\subsection{Text Input}
			\label{ss:TextInput}
			An example is given below:

\begin{lstlisting}[frame=trBL, breaklines=true, language = bash]		
while :
do
	# We will need to prompt the user for the name of the component to create.
	echo -e -n "${GREEN}Enter a name for the component to be created: (No spaces) ${NO_COLOUR}"
	read
	echo -e "" 
	# Check the user actually entered something.
	if [ -z "$REPLY" ]; then
		# We'll just prompt again.
		continue
	# Check to see if this name is already taken.
	elif [ -d "$TCPATH/src/$REPLY" ]; then
		# The name provided is already taken.  Prompt the user to choose an available name.
		echo -e "${RED}The component '$REPLY' already exists.  Please choose another name.\n${NO_COLOUR}"
	else
	# The name is probably legit, so we move on.
		COMPONENT=$REPLY
		break
	fi
done
\end{lstlisting}

\subsection{Yes/No Prompts}
			\label{ss:YNPrompts}
			
\begin{lstlisting}[frame=trBL, breaklines=true, language = bash]
echo -e -n "${GREEN}Do you wish to make this the current component? (Y/N)${NO_COLOUR}"
read -n 1
echo -e "\n" # NOTE - This is because the read command won't put a newline after it reads a character.	

# If they responded YES, then set the current component.	
if [[ $REPLY =~ ^[Yy]$ ]]; then
	# We want to set this as the current component.
	sed -i "s^\(tc_curr *= *\).*^\1$COMPONENT^" $TCPATH/$USER_CONFIG_FILE
fi			
\end{lstlisting}

\section{Spaced Lists}
			\label{s:SpacedLists}
			A lot of the time, names (such as those of files or components) will be stored in variables. These are very useful for things such as copying files, or iterating through a \texttt{for} loop with each element being used once. 
			For this reason, we must make sure that whenever we store an array of names in a variable, they are correct. This means that none of the names can have spaces, for this would treat one name as two. 
			Also, it depends on its use, but sometimes, for example with the \texttt{ls} command, a newline separated list can result in an error. This is particularly relevant for entries into a makefile. Whenever practical, try to make sure these lists separate their elements with spaces rather than newline characters. 
			This is shown in the \texttt{build} script:
			
\begin{lstlisting}[frame=trBL, breaklines=true, language = bash]
C_FILES=$(ls $TCPATH/$TMP_SRC_DIR/$COMPONENT/*.c 2>/dev/null | sed ':a;N;$!ba;s/\n/ /g')
\end{lstlisting}

			In this case the sed function at the end removes all newline characters and replaces them with spaces.

\section{Style}
			\label{s:Style}
			Try to keep the style of coding similar to that already established. 
\begin{itemize}
\item Comments should appear above the relevant line, rather than to the right. Comments should be frequent and verbose.
\item Variables should be in all caps, with underscores to separate words in one variable.
\item Functions are lower case, with underscores to separate words.
\item Tabbing should follow standard coding practice.
\end{itemize}
			

\pagebreak
			
\chapter{Script Descriptions}
			\label{C:ScriptDescriptions}
			In this section, some of the more essential and complex scripts are described in detail, with an idea of how they work, and what aspect of the overall picture they relate to.

\section{Script: VFstart}
			\label{s:VFstart}
			Location: \texttt{ValleyForge/bld/VFstart}
		
\subsection{Summary}
			\label{ss:VFstartSummary}		
			VFstart is the main interface to the toolchain when not building a component.
			
\subsection{Prologue}
			\label{ss:VFstartPrologue}
			It starts off, like all the scripts, by finding the file location of the toolchain. It then defines some constants, in this case locations of other relevant files, and imports the colour scheme.

\subsection{Functions}
			\label{ss:VFstartFunctions}			
			\texttt{Usage} -- This function is simply a print to terminal. It shows the user the usage of the script, particularly the parameters that can be passed to it and what they do. 
			
\subsection{Script Proper}
			\label{ss:VFstartScriptProper}
			
\subsubsection{Resetting Fields}
			\label{sss:VFstartResettingFields}

			Resetting fields is a safety measure to ensure that no wrong parameters are passed around, and an error can be picked up easily if the variables are empty. Simply entering:
			
\begin{lstlisting}[frame=trBL, breaklines=true, language = bash]				
VARIABLE=			
\end{lstlisting}				

clears the variable, meaning that if it is not reassigned, an empty value will be assigned to it. It is easy to check for a variable being empty, whereas if it has the wrong value assigned to it, this is not as easy to pick up.

\subsubsection{Parsing Parameters}
			\label{sss:VFstartPassingParameters}
			The script then parses command line parameters, as described in section \S\ref{s:ScriptParameters}.
			The parameters for VFstart are:
			
\begin{itemize}

\item \texttt{\textbf{help}}. A standard option for any script, this prints out the usage text to the terminal

\item \texttt{\textbf{block}}. If the user specifies the parameter ``Block Mode'', then once the user has performed one action, it will prompt for another, rather than the default action of exiting the interface once an action (such as creating a new component) has been performed.
\end{itemize}

\subsubsection{Loop}
			\label{sss:VFstartLoop}
			The while loop serves to enable the ``Block Mode'', where once one task is completed, another can begin.
			The variable ``REDRAW\_TOP'' is used to bring the user back to the top level menu if set at certain points during the script. The script checks the state of the variable and goes back to the starting interface if set. It is reset when the while loop starts again.
			
\subsubsection{User Config File}
			\label{sss:VFstartUserConfigFile}
			
			The toolchain then does some basic checks to discern the nature of the environment. It checks to see if a user config file is present, creates one if there isn't, and asks the user to fill out their info if it hasn't been set. 
			Currently, all that this does is put their name at the start of source files when they are created, but future functionality could be added (like company, aspect working on).
			
\subsubsection{Active Component}
			\label{sss:VFstartActiveComponent}
			
			The script checks the user config file stored in \texttt{var/config.cfg} to set what the field \texttt{tc\_curr} is set to. If it does not correspond to a folder in the src directory, the user is warned and advised to set a component to be active, and told how to create new components.
			
\subsubsection{Opening Prompt}
			\label{sss:VFstartOpeningPrompt}
			The script gives the user four options, and the user selects one by pressing the key corresponding to the underlined letter in the word.
\begin{itemize}
\item \textbf{Create}. To create a new component, library, or files. See section \S\ref{s:createComponent}.
\item \textbf{Edit}. To edit user info, active a component, or edit toolchain settings.
\item \textbf{Delete}. To delete components or libraries.
\item \textbf{Quit}. Quits the script.
\end{itemize}			
			 
			
\subsubsection{Create Menu}
			\label{sss:VFstartCreateMenu}
			The script gives the user four options:
\begin{itemize}
\item \textbf{Create Component}. This runs the create\_component script located in bld/create. A description of the script can be found in section \S\ref{s:createComponent}.
\item \textbf{Create Files}. This runs the create\_files script located in bld/create. A description of the script can be found in section \S\ref{s:createFiles}.
\item \textbf{Create Library}. This runs the create\_library script located in bld/create.
\item \textbf{Back}. Go back one level.
\end{itemize}				
			


\subsubsection{Delete Menu}
			\label{sss:VFstartDeleteMenu}
			The script gives the user three options:
\begin{itemize}
\item \textbf{Delete Component}. This runs the delete\_component script located in bld/delete.
\item \textbf{Create Library}. This runs the delete\_library script located in bld/delete.
\item \textbf{Back} Go back one level.
\end{itemize}			


\subsubsection{Edit Menu}
			\label{sss:VFstartEditMenu}
			This menu gives the user four options:
\begin{itemize}
\item \textbf{Activate Component}. This lists the existing components (determined by folders existing in the \texttt{src} directory) and lets the user choose one to be the active component.
\item \textbf{Reconfigure Active Component}. This runs the create\_component script with the name of the active component passed as a parameter.
\item \textbf{Edit User Config}. This runs the script \texttt{edit\_user\_config}, located in \texttt{bld/edit}. 
\item \textbf{Back} Go back one level.
\end{itemize}					

\subsubsection{Epilogue}
			\label{sss:VFstartEpilogue}
			The running of the script is checked for errors. If there are any errors and the script is in Block mode, then the script will ask for user confirmation before proceeding, because the top of the while loop clears the screen, erasing the error messages that the user would want to see.
			
			The script checks the REDRAW and BLOCKMODE variables and reruns the while loop to the top level menu or exits the script as desired.
			
\section{Script: build}
			\label{s:build}
			The build script gets run whenever the user wants to compile their component into a binary to upload to their device.
			
\subsection{Summary}
			\label{ss:buildSummary}
			Whenever the user wants to compile their component, they compile it using the build script. It is set up such that one only has to run this script, perhaps with a few parameters, to successfully build the component. With no parameters passed, it simply builds the active component.
			The paramaters that can be passed to this script currently are:
			
\begin{itemize}
\item \texttt{\textbf{help}}. This prints the usage message, informing the user how to use the script.
\item \texttt{\textbf{all}}. This builds all the components, one after the other. If one component fails to build, the others will still be attempted.
\item \texttt{\textbf{retain}}. Often, to debug preprocessing errors, or inclusion errors, it is useful to look at the code that is actually being compiled, as it may differ from how it is found in the source folders. The retain functionality does not delete the files from the tmp directory, so that the user may look at them. They do, however get deleted straight away next time the build script is run.
\item \texttt{\textbf{loader}}. This builds the corresponding bootloader as well, if there is one. It compiles in a subdirectory of \texttt{tmp\_src} called bootloader, and the resulting \texttt{hex} file is called COMPONENT\_NAME\_bootloader.hex and also placed in the \texttt{bin} directory.
\item	\texttt{\textbf{postpack}} After the build operation has completed, this option will pack up (remove) any compilers or other tools which needed to be unpacked as part of the build process.  This will save space and tidy things up, but means that the tools will need to be unpacked again next time, so subsequent build runs may be slower than they would be otherwise.
\item \texttt{\textbf{name}}. This will build the specified component only.
\item \texttt{\textbf{nohal}}. While HAL aspects are not yet complete, it can be useful to build components without any use of the HAL. This makes sure the HAL files are not put into the build directory.
\end{itemize}

\subsection{Prologue}
			\label{ss:buildPrologue}
			As per usual, the script first finds out where it is located, and the absolute path of the toolchain, defines a few constants (mainly file locations), and sources the colour scheme (see \S\ref{s:Colours}), as well as the preprocessing functions (see more about the preprocessing script in section \S\ref{s:Preprocess})
			
\subsection{Functions}
			\label{ss:buildFunctions}
			The build script has quite a few functions...
			
\subsubsection{makeavr}
			\label{sss:buildMakeavr}
			This function is responsible for compiling and linking avr files into a hex binary input.\newline
			First it copies the makefile to be used from its source, in this case \texttt{res/avr}.\newline
			It then checks to see if a linker script has been specified for this particular build configuration. If so, it copies it from the location specified in the build configuration to the \texttt{tmp\_src/component} directory, used for compiling files.\newline \newline
			The build script has already copied all relevant source files to the build directory from various sources. Now the \texttt{makeavr} function creates a list of all the source files to be built, and their type.\newline
			Next, the function edits the makefile, replacing strings such as ``BUILD\_INSERTS\_C\_FILES\_HERE'' with the relevant entries using \texttt{sed}. This is contrast to the replacements that follow, which erase everything to the right of the ``='' and therefore can't be used to add to a line. If a variable is empty, the string is simply replaced with nothing.\newline \newline
			The makefile is then run. All makefile output is printed to the screen. If the makefile returns with an error, the whole script exits and doesn't attempt to do anymore, else, if it compiles successfully, it copies the binary over to the \texttt{bin} folder.
			
\subsubsection{make\_avr\_bootloader}
			\label{sss:buildMakeavrbootloader}
			This function does the same as the above in section \S\ref{sss:buildMakeavr} except for the avr\_bootloader. Different make functions are used for each because the makefiles are different and need different fields replaced, however, it probably wouldn't be too hard to combine all these make functions together.
			
\subsubsection{makeavr32}
			\label{sss:buildMakeavr32}
			Same as above, see section \S\ref{sss:buildMakeavr}.
			
\subsubsection{makearm}
			\label{sss:buildMakearm}
			Same as above, see section \S\ref{sss:buildMakeavr}.
			
\subsubsection{inflate\_avr32cc}
			\label{sss:buildInflateavr32cc}
			The AVR32 compiler is kept in a tarball, and is inflated whenever required. This function simply inflates the tarball, which contains the relevant binaries (and header files).
			
\subsubsection{get\_rtos\_files}
			\label{sss:buildGetrtosfiles}
			This function retrieves the relevant rtos files for respective build configurations and places them in the build directory.
			
\subsubsection{get\_suitable\_libs}
			\label{sss:buildGetsuitablelibs}
			If the user has specified a library to be included in the component, this function retrieves the relevant files and copies them into the build directory. In the tmp directory these files are placed in a subdirectory called ``lib''.
			
\subsubsection{get\_hal}
			\label{sss:buildGethal}
			Retrieves the relevant HAL files and places them in the build directory (\texttt{tmp/tmp\_src/hal}). It first checks to see that all necessary parameters are present in order to implement HAL features. Then it copies the base HAL files, including semaphores, a HAL layout, and a target config file, into \texttt{tmp/tmp\_src}. \newline
			It then checks which HAL elements are specified to be included in the corresponding build configuration, and copies these over too. They are comprised of a header file and a target specific c++ file. All HAL files are preprocessed to ensure the correct values for the build configuration.
			
\subsubsection{usage}
			\label{sss:buildUsage}
			This prints a message to the terminal outlining how to use the script, which parameters can be passed to it. To see the parameters, see the build summary in section \S\ref{s:build}. 

			
			
\subsection{Script Proper}
			\label{sss:buildScriptProper}
			First everything in the \texttt{tmp/tmp\_src} directory is deleted. The parameters are then processed. How this process works and can be altered is shown in section \S\ref{s:ScriptParameters}.\newline
			If the ``all'' parameter is not specified, the config file is checked to see what the active component is. The variable ``\texttt{NAME}'' is set to have the name of the component.\newline
			If the ``all'' parameter is specified, then the variable ``\texttt{NAME}'' is set as the names of all components, space separated.\newline
			The \texttt{build\_configs.cfg} is sourced. Note that at this stage no code is run here, since it is comprised entirely of functions that need to be called.\newline
			A \texttt{for} loop is run, iterating through all components to be built. So usually just once.\newline
			The component config file is located. If it is not found, the user is prompted to create one. Else it is sourced, i.e its variable values read in. Any build config variables are cleared by running the \texttt{clear\_bconf\_vars} script. If you add any variables to any of the build configurations, make sure this script clears them, as otherwise values may be retained and lead to errors.\newline
			The build script then finds and copies all the user source files into the build directory \texttt{tmp/tmp\_src}, if there are none it stops building the component. All these files are then preprocessed. See a detailed description of the preprocess script in section \S\ref{s:Preprocess}.
			The two functions, \texttt{get\_suitable\_libs} (see section \S\ref{sss:buildGetsuitablelibs}) and \texttt{get\_hal} (see section \S\ref{sss:buildGethal}) are then run.
			
			The actual make function is then run. These functions are described above from section \S\ref{ss:buildFunctions}.			
			A check to see if everything ran without errors is then executed, and the bootloader compiled if the loader flag was set. This is done pretty much the same way as the application.
			The build directory is then cleared (or not if the retain flag is set) and the script finishes.
			

\section{Script: Create\_component}
			\label{s:createComponent}
			This script is used to create new components for the user's version of the toolchain.
			
\subsection{Summary}
			\label{ss:crecompSummary}
			This script is generally called from within other scripts rather than being called by itself, although it can be called by itself, with no parameters. It is responsible for creating new components, including building a source directory, making the config file, setting up the bootloader, and calling the \texttt{create\_files} (section \S\ref{s:createFiles}) script, if requested to create source files.
			The script has the following available parameters:
			
\begin{itemize}
\item \texttt{help}. Runs the usage function, showing the usage of the script.
\item \texttt{batch}. Runs the script in batch mode. This means there are no prompts, and all parameters must be specified in the script parameter set.
\item \texttt{component}. For batch mode, specifies the name of the component.
\item \texttt{subsystem}. For batch mode, specifies the name of the subsystem.
\item \texttt{target}. For batch mode, specifies the name of the target.
\item \texttt{platform}. For batch mode, specifies the platform for the component.
\item \texttt{loader}. For batch mode, specifies the bootloader for the component.
\end{itemize}
			
\subsection{Prologue}
			\label{ss:crecompPrologue}
			The script first finds out where it is located, and the absolute path of the toolchain, defines a few constants (mainly file locations), and sources the colour scheme (see \S\ref{s:Colours}), as well as the preprocessing functions (see more about the preprocessing script in section \S\ref{s:Preprocess})
			
\subsection{Functions}
			\label{ss:crecompFunctions}
			
\subsubsection{choose\_target}
			\label{sss:crecompChooseTarget}
			This searches the ``\texttt{build\_configs.cfg}'' file for application build configurations, extracts the target name from each, and lists them in a list to select (selecting method can be found in section \S\ref{ss:EnumeratedList}). 
			The variable ``\texttt{TARGET}'' is assigned the name of the target that the user chooses.
			
\subsubsection{choose\_platform}
			\label{sss:crecompChoosePlatform}
			This searches the ''\texttt{build\_configs.cfg}'' file for application build configurations	for the specified target. It then lists the platforms available for the target, the user selects one, and it is recorded in the variable ``\texttt{PLATFORM}''. The ``cut'' function takes the names of all the functions in the \texttt{build\_configs.cfg} file, and cuts everything else away leaving only the name of the platform.
			
\subsubsection{choose\_bootloader}
			\label{sss:crecompChooseBootloader}
			This searches the ''\texttt{build\_configs.cfg}'' file for bootloader build configurations for the specified target. If there are bootloader build configurations available, the user selects one from an enumerated list, else this is skipped.
			
\subsubsection{run\_bconf\_specific\_config}
			\label{sss:crecompRunBconfSpecificConfig}
			Some build configurations require extra setup specific to them, so for build configurations where this is the case, an extra script kept in the \texttt{res} directory is run. 
			This function checks to see if the build configuration specifies any extra script to run, and runs it if that is the case.
			
\subsubsection{run\_bload\_specific\_config}
			\label{sss:crecompRunBloadSpecificConfig}	
			Some build configurations require extra setup specific to them, so for build configurations where this is the case, an extra script kept in the \texttt{res} directory is run. For example for the AVR bootloader, an extra script is run asking the user to specify which pins are to be used by the bootloader for the leds or for the input pin.\newline
			
\subsubsection{usage}
			\label{sss:crecompUsage}
			
			
\subsection{Script Proper}
			\label{ss:crecompScriptProper}
			
			Fields are reset, and parameters parsed (see section \S\ref{s:ScriptParameters} for how this is done). \newline
			The user is asked to provide a component name. This name is tested for legitimacy, by checking to see it doesn't have any spaces, is not empty, does not use any illegal characters, and is not a reserved word. Then the same is done for the subsystem name.\newline
			The user then chooses the target, platform, and bootloader. Then the component's source directory is created, and the component config file placed in this directory.\newline
			The user is asked whether they want to make this new component the active component. If so, the value in the \texttt{config.cfg} file in \texttt{var} is edited.\newline
			Next they are asked whether they want to create files for the new component. If so, the \texttt{create/create\_files} script is run (see section \S\ref{s:createFiles}).
			
\section{Script: Create Files}
			\label{s:createFiles}
			This takes some template source files, copies and edits them, and places them into a component's source directory.
	
\subsection{Summary}
			\label{ss:crefileSummary}
			The \texttt{create\_files} script uses the template files located in \texttt{res/templates} (see section \S\ref{ss:res/templates}). When the user specifies new files to be created, it edits values, such as subsystem, user info, component name, target name, and lays out a style to follow. It creates a header file and a source file (can be C or C++).
			The parameters are
			
\begin{itemize}
\item \texttt{help}. Runs the usage function, showing the usage of the script.
\item \texttt{batch}. Runs the script in batch mode. This means there are no prompts, and all parameters must be specified in the script parameter set.
\item \texttt{name}. Specifies the name of files to be created.
\item \texttt{component}. For batch mode, specifies the name of files.
\item \texttt{type}. For batch mode, specifies the of file (C or C++).
\end{itemize}
			
\subsection{Prologue}
			\label{ss:crefilePrologue}
			Starts off finding its own file location and that of the toolchain. it then defines some constants, in this case paths to the template files, as well as the user config file and the \texttt{build\_configs.cfg} build configurations file. It sources the colours and preprocessing functions.
			
\subsection{Functions}
			\label{ss:crefileFunctions}
			
\subsubsection{Usage}
			\label{sss:crefileUsage}
			This prints a message to the terminal which tells the user how to use the script, listing its parameters can giving a summary of its purpose and usage.
			
\subsection{Script Proper}
			\label{ss:crefileScriptProper}
			A few fields are reset, and then the parameters are parsed (see section \S\ref{s:ScriptParameters}). \newline
			Assuming we aren't in batch mode, the user is prompted for a name for the pair of files. If a component has been specified, its config file will be read.\newline
			Else the user config file is read, the name of the active component extraced, and the active component used to create files for.\newline
			The user chooses the source file type (C or C++ so far) and the script does a few error checks.\newline
			Then the script reads in the build configurations, and tries to find a matching one. If one is found, the appropriate files are preprocessed using the \texttt{preprocess\_template} script. There are four different main templates covering the possibilities of FreeRTOS or Bare Metal, and C or C++.\newline
			If the component uses FreeRTOS, and no \texttt{FreeRTOSConfig.h} file exists in the source directory, then it is copied from its location in res. FreeRTOS requires this header file to run.
			
			
\section{Script: preprocess}
			\label{s:Preprocess}
			This script is used to edit generic files to change them to application specific instances.

\subsection{Summary}
			\label{ss:preprocessSummary}
			This script, like \texttt{build\_configs.cfg}, has no script proper, and is just some functions. Each function corresponds to a different file being processed. In each file are different fields to be edited. \newline
			
			The functions go through the various files, replacing strings such as ``<<<TOOLCHAIN\_INSERTS\_C\_FILES\_HERE>>>'' with the relevant entries, or replacing everything to the right of a specified ``=''. If a variable is empty, the string is simply replaced with nothing.\newline
			The only parameter passed to any of the functions is the path of the file to be preprocessed.\newline
			
			Each function uses sed to find and replace. Sed is hard to use, just find examples in these scripts and modify them if you are not familiar with sed.
			
\section{Script: buildconfigs.cfg}
			\label{s:buildConfigs}
			The build configs are simply functions that are called, assigning values to specified fields. The variables that are mostly assigned are:
			

\begin{itemize}
\item \texttt{BLOAD\_NAME or BCONF\_NAME}: The string that is the display name for this build configuration
\item \texttt{C\_COMPILER}: The compiler used to compile C files, i.e what is placed in the makefile in this space.
\item \texttt{P\_COMPILER}: The compiler used to compile C++ files, i.e what is placed in the makefile in this space.
\item \texttt{OBJCOPY}: The program used to convert output files.
\item \texttt{OBJDUMP}: The program used to inspect object files and their properties.
\item \texttt{SIZE}: The program used to measure the size of output files.
\item \texttt{MAKEFUNCTION}: The function within build used to build this configuration.
\item \texttt{MAKEFILE}: The path to the makefile used to build this configuration.
\item \texttt{SOURCEPATH}: The path to the source files for this configuration (not valid for all configurations, mainly just for bootloader and FreeRTOS).
\item \texttt{MCU\_CODE}: The mcu code that the compiler receives to know which target to compile for.
\item \texttt{IOHEADER}: The name of the io file to include.
\item \texttt{STDINTHEADER}: The name of the stdint file to include.
\item \texttt{BOOTSTART}: Only valid for bootloader configs, the start address of the bootloader.
\item \texttt{PORT\_COUNT}: All the gpio ports on the microprocessor being used.
\item \texttt{BLOAD\_SPECIFIC\_CONFIG or BCONF\_SPECIFIC\_CONFIG}: The location of the script to be executed when a new instance of a component with this build configuration is created.
\item \texttt{ACTIVE\_MODULE}: For bootloaders, the module (isp,can,i2c) used to transfer data during bootloading.
\item \texttt{MAX\_EEPROM\_ADDRESS}: Each target has a limited amount of EEPROM memory, this indicates how large it is and its max address.
\item \texttt{LINKSCRIPT}: If the build configuration needs a special linker script, this shows its path.
\item \texttt{FREERTOS\_CONFIG\_FILE}: The path to the FreeRTOSConfig.h file for this configuration.
\item \texttt{CFLAGS}: Flags to add to the C compiler.
\item \texttt{PFLAGS}: Flags to add to the C++ compiler.
\item \texttt{AFLAGS}: Flags to add to the Assembler compiler.
\item \texttt{HAL\_HEADER\_PATH}: Path to the HAL headers.
\item \texttt{HAL\_SOURCE\_PATH}: Path to the HAL C++ files specific to this build configuration.
\item \texttt{HAL\_EN\_LIST}: A list of the HAL modules to be used in this build configuration.
\item \texttt{BCONF\_SPECIFIC\_RTOS\_PATH}: When FreeRTOS is being used, the path to the port-specific FreeRTOS files.
\item \texttt{BCONF\_COMMON\_RTOS\_PATH}: When FreeRTOS is being used, the path to the commmon FreeRTOS files.
\end{itemize}
			
			
\chapter{HAL}
			\label{C:HAL}
			The ValleyForge Hardware Abstraction library is implemented to give a common interface to various microprocessors and make code portable between devices. \newline
			At this stage seven modules' interface structure has been completed: \texttt{gpio.h}, \texttt{mem.h}, \texttt{tc.h},\texttt{watchdog.h}, \texttt{usart.h}, \texttt{spi.h} and \texttt{i2c.h}. Five of these modules (\texttt{gpio.h}, \texttt{mem.h}, \texttt{tc.h}, \texttt{watchdog.h} and \texttt{usart.h}) have been implemented for one microprocessory only, the ATmega2560. The interface for the remaining modules has been constructed but the implementation has not been fully implemented to a functional state.\newline 
			As well as this, there is a general HAL file which contains global interrupt enable/disable, and which links to the \texttt{semaphore} source files, and a \texttt{target\_config.h} header file, which has a lot of ``\texttt{\#define}''s outlining different kinds of constants for each target.\newline
			The HAL modules are implemented in C++.\newline
			Although only the ATmega2560 modules have been implemented, the AVR architectures will all use the same implementation file, by using the AVR preprocessor to select which parts of code are relevant to the specific microcontroller.
			
\section{Modules}
			\label{s:HALModules}
			The currently implemented modules:
			
\begin{itemize}
\item \textbf{\texttt{gpio}}. The gpio module uses a two-dimensional array of binary semaphores to hold assign GPIO pin usage. A user uses the \texttt{pin\_grab} function to be given a class instance of a pin, which they can then manipulate until they relinquish hold on it again. 
\item \textbf{\texttt{tc}}. The Timer/Counter Module does not use semaphores, due to the scarcity of timers and then need to use one timer for multiple applications. The user is given an instance of a timer class and is able to use this to manipulate the timers. Due to timers varying greatly between different microprocessors, this module is large and messy.
\item \textbf{\texttt{mem}}. The memory module writes to EEPROM. For the AVR architectures it is really no more than a layer between the \texttt{avr/eeprom} module and the user.
\item \textbf{\texttt{watchdog}}. This implements simple watchdog timers.
\item \textbf{\texttt{usart}}. The USART module caters for serial communication between the target and a host of other devices. The implementation of the module allows a user to initialise a USART channel with a host of options, transmit and receive data as well as relinquish control of it.
\item \textbf{\texttt{spi}}. The SPI module will allow the user to access and use the target's SPI associated hardware. The present interface allows for initialisation of the hardware and transfer of data in both blocking and non-blocking modes.
\item \textbf{\texttt{i2c}}. The I$^{2}$C module will provide the necessary functionality to utilise the I$^{2}$C communication protocol with the target microprocessors. The interface currently assumes the target to be the master of the I$^{2}$C network for all communication with slave devices.
\end{itemize}

\section{\textbf{\texttt{hal}} Module}
			\label{s:HALhalModule}
			
			The \textbf{\texttt{hal}} module provides required functionality for all the implemented software modules listed in \S\ref{s:HALModules}.
			
			For future porting of the HAL to other microprocessors, it will be necessary to edit the enumerated \textbf{\texttt{port\_t}} and \textbf{\texttt{pin\_t}} listings to cater for specific hardware.
			
			In addition, useful constants such as \textbf{\texttt{PORT\_MULTIPLIER}}, \textbf{\texttt{P\_OFFSET}} and the \textbf{\texttt{port\_offset}} list aid generalised access to memory registers in other modules. These have been tailored for the ATmega2560 and may or may not be correct for future targets.

\section{\textbf{\texttt{gpio}} Module}
			\label{s:HALgpioModule}
			The \textbf{\texttt{gpio}} module allows the user to interact with the target's General Purpose I/O (GPIO) hardware. The module caters for the the ability to set the input/output functionality of a GPIO pin, set the pin's output logic, read the logic value of the pin and attach or disable interrupt functions to pins that support this functionality.

\subsection{\textbf{\texttt{grab}} Function}
			\label{ss:HALgrabFunction}
			The \textbf{\texttt{grab}} function allows the user to create an instance of a GPIO pin that they can then use through the available methods in the \textbf{\texttt{gpio}} class. 
			
			This function takes as an argument a \textbf{\texttt{gpio\_pin\_address}} which must be of a valid GPIO pin on the microprocessor. Internally, the function proceed initialises the GPIO implementations the first time a GPIO pin is `grabbed' by assigning the semaphore pointer within each GPIO to the semaphore that belongs to that particular pin. The function then attempts to procure the GPIO semaphore for the GPIO pin addressed by the user. If this is successful, a pointer to the implementation is assigned and the function will return non-NULL.
			
			Valid GPIO pin addresses for each microprocessor are made up of a specific pin and port and are contained within enumerated lists found in the \textbf{\texttt{hal.h}} file. These lists should be modified when porting to different targets by utilising the pre-processor to selectively compile different lists depending on the requirements of the new target.  			

\subsection{\textbf{\texttt{set\_mode}} Function}
			\label{ss:HALset_modeFunction}
			This function enables the user to set the desired directionality of the GPIO pin attached to the instance created. The options available for this function to supply as an argument are either \textbf{\texttt{INPUT}} or \textbf{\texttt{OUTPUT}}.
			
			In the AVR implementation of the \textbf{\texttt{gpio}} module, the \textbf{\texttt{set\_mode}} selectively edits the microprocessor's Data Direction Register (\textbf{\texttt{DDR}}) for the port of the GPIO pin to either set it as an input or output. The memory address of the register is accessed by a simple arithmetic process on the address of the GPIO pin. There are three constants that are used in this process:
			
\begin{itemize}
\item \textbf{\texttt{PORT\_MULTIPLIER}}. This constant simply caters for the spacing in memory between GPIO control registers of different ports.
\item \textbf{\texttt{P\_OFFSET}}. This accounts for the 'jump' in memory addresses between GPIO control registers of \textbf{\texttt{Port G}} and \textbf{\texttt{Port H}}. Note, this is ATmega 2560 specific and may be different or not required on other platforms.
\item \textbf{\texttt{P\_MODE}}. Part of an enumerated list that also includes \textbf{\texttt{P\_READ}} and \textbf{\texttt{P\_WRITE}}. These values allow the \textbf{\texttt{PIN}}, \textbf{\texttt{DDR}} and \textbf{\texttt{PORT}} registers to be accessed sequentially.
\end{itemize} 
			
			The editing of the \textbf{\texttt{DDR}} register is accomplished by accessing the register's memory space through the \textbf{\texttt{\_SFR\_MEM8}} or \textbf{\texttt{\_SFR\_IO8}} macros and specific register addresses found in the AVR \textbf{\texttt{iomxx0\_1.h}} header file.
			
			This function returns \textbf{\texttt{0}} always.
			
\subsection{\textbf{\texttt{write}} Function}
			\label{ss:HALwriteFunction}
			The \textbf{\texttt{write}} function gives the ability to output a logic value on a GPIO pin.
			
			The function accepts a value of type \textbf{\texttt{gpio\_output\_state}} and consists of one of the following:
			
\begin{itemize}
\item \textbf{\texttt{O\_LOW}}. Sets the GPIO pin to logic \textbf{\texttt{0}}.
\item \textbf{\texttt{O\_HIGH}}. Sets the GPIO pin to logic \textbf{\texttt{1}}.
\item \textbf{\texttt{O\_TOGGLE}}. Toggles the GPIO pin's logic irrespective of its current value.
\end{itemize}

The \textbf{\texttt{write}} function accesses the \textbf{\texttt{PORT}} registers through the same macros and constants as those discussed in \S\ref{ss:HALset_modeFunction}.

This function returns \textbf{\texttt{0}} always.

\subsection{\textbf{\texttt{read}} Function}
			\label{ss:HALreadFunction}
			
			This function accepts no input arguments but instead returns a value indicating the logic of the GPIO pin it is called with. These values will be one of the following \textbf{\texttt{gpio\_input\_state}} enumerated values:
			
\begin{itemize}
\item \textbf{\texttt{I\_LOW}}. Logic \textbf{\texttt{0}}.
\item \textbf{\texttt{I\_HIGH}}. Logic \textbf{\texttt{1}}.
\item \textbf{\texttt{I\_ERROR}}. GPIO port out of bounds.
\end{itemize}

			Once again, this function accesses the AVR's \textbf{\texttt{PIN}} registers through the use of macros and constants discussed in \S\ref{ss:HALset_modeFunction}.
			
\subsection{\textbf{\texttt{attach\_interrupt}} Function}
			\label{ss:HALattach_interruptFunction}
			
			The \textbf{\texttt{attach\_interrupt}} function allows a user to register a function to be executed when an associated event occurs on the implemented GPIO pin.
			
			The function accepts a pointer to a user defined ISR function along with a mode that represents a value from the \textbf{\texttt{interrupt\_mode}} enumerated list found in \textbf{\texttt{gpio.h}}. 
			
			The GPIO address of the implemented pin is evaluated to determine whether it is capable of provinding Pin Change Interrupt request (\textbf{\texttt{PCINT}}) or External Interrupt (\textbf{\texttt{EXINT}}) functionality.
			
			The attach function proceeds to attempt to procure a semaphore for the interrupt and set the appropriate bits within the interrupt mask registers depending on which pin is implemented. If the pin is capable of (\textbf{\texttt{EXINT}}) operation then the mode provided is used to configure the external interrupt control registers. The function also places a pointer to the user defined ISR within the private global interrupt function array.
			
			The function returns one of three values. (\textbf{\texttt{GP\_SUCCESS}}) indicates successful completion of the function, (\textbf{\texttt{GP\_ALREADY\_TAKEN}}) conveys the semaphore for the requested interrupt is already held elsewhere and (\textbf{\texttt{GP\_OUT\_OF\_RANGE}}) implies the pin that interrupt has been attached to does not have an interrupt vector available to it.
			
\subsection{\textbf{\texttt{disable\_interrupt}} Function}
			\label{ss:HALdisable_interruptFunction}
			
			This function removes the association between a registered application ISR and the GPIO event which triggers it. 
			
			The function is performs this role by clearing the bits in the interrupt mask register which cause the ISR to fire, and relinquishing the semaphore for the interrupt that was previously procured. The pointer to the user ISR in the function pointer array is replaced with NULL too.
			
\section{\textbf{\texttt{tc}} Module}
			\label{s:HALtcModule}
			The \textbf{\texttt{tc}} module provides functionality for the hardware timers and counters. This abstracted control gives the user the ability to initialise, configure and run these elements in a variety of modes.
			
\subsection{\textbf{\texttt{grab}} Function}
			\label{ss:HALtcgrabFunction}
			The grab function instantiates a timer instance that the user can then use in conjunction with the associated methods in the class to perform the functions they wish.
			
			A timer counter number is required to be given to the \textbf{\texttt{grab}} function to proceed with initialisation. This timer counter number is required to be one of an enumerated list found in the \textbf{\texttt{tc\_number}} definition within the \textbf{\texttt{tc.h}} file. 
			
			The first time the \textbf{\texttt{grab}} function is called, a process of initialising all the timer implementations is carried out. This includes assiging the relevant GPIO addresses and semaphores. The information used for this initialisation is found as pre-processor definitions within the \textbf{\texttt{tc.cpp}} file.
			
			Presently, the \textbf{\texttt{tc}} module does not employ semaphores to manage or control access to each of the independent timers. However, this is a possible feature that could be included in the future.
			
			The \textbf{\texttt{grab}} creates a pointer to the implementation requested by the user and returns a timer instance for future use. 
			
\subsection{\textbf{\texttt{set\_rate}} Function}
			\label{ss:HALtcset_rateFunction}
			
			The \textbf{\texttt{set\_rate}} function performs a number of important roles in configuring a timer counter instance.
			
			This function requires the user to provide a \textbf{\texttt{rate}} value which is a custom structure made up both fields to indicate the desired clock source and prescalar for the timer counter. The options for these values are found in the enumerated lists \textbf{\texttt{tc\_clk\_src}} and \textbf{\texttt{tc\_prescalar}} respectively, within the \textbf{\texttt{tc.h}} file.
			
			The function assigns the relevant register addresses to the \textbf{\texttt{registerTable}} field within the implementation class as well as the user defined prescalar setting and the resolution of the timer.
			
			A value of \textbf{\texttt{0}} is returned to indicated success, otherwise if an invalid prescalar setting for the timer chosen is supplied, \textbf{\texttt{-1}} is returned instead.
			
\subsection{\textbf{\texttt{load\_timer\_value}} Function}
			\label{ss:HALtcload_timer_valueFunction}
			
			In order to load a specific value into the timer counter register this function should be called. An input value should be provided and an appropriate template employed depending on the size of the input value and the size of the timer counter register.
			
			Internally, the function applied the value provided to the register at the pre-defined memory address. This process varies depending on whether the timer implementation relates to a 16-bit or 8-bit timer, and whether the register occupies a higher memory address that requires the use of the \textbf{\texttt{\_SFR\_MEM8}} macro.
			
			This function returns \textbf{\texttt{0}} always.
			
\subsection{\textbf{\texttt{get\_timer\_value}} Function}
			\label{ss:HALtcget_timer_valueFunction}
			
			This function provides the ability to read the current value of the timer counter. No input variables are required but the user must anticapate the size of the returned timer value and apply an appropriate template to suit this.
			
			Practically, this function varies little from that in \S\ref{ss:HALtcload_timer_valueFunction}. It simply returns the value from the register that it accesses via the appropriate macro.
			
\subsection{\textbf{\texttt{start}} Function}
			\label{ss:HALtcstartFunction}
			
			As the name suggests, this function starts the implemented timer counter by loading the pre-determined prescalar settings into the appropriate registers. Henceforth, the timer counter will begin its operation according to the user defined initialisation settings.
			
			Depending on the timer implemented, this function will call one of three secondary functions, namely \textbf{\texttt{start\_timer0}}, \textbf{\texttt{start\_16bit\_timer}} or \textbf{\texttt{start\_timer2}}. The reasoning behind this is the variety of applicable prescalar settings and registers that are required to be accessed varies depending on the exact timer used.
			
			This function returns \textbf{\texttt{0}} if the timer is supplied a valid prescalar, otherwise it will return \textbf{\texttt{-1}}. However, the prescalar setting is checked by the \textbf{\texttt{set\_rate}} function so this eventuality should not occur.
			
\subsection{\textbf{\texttt{stop}} Function}
			\label{ss:HALtcstopFunction}
			
			The \textbf{\texttt{stop}} function causes the operation of the timer counter to cease by clearing the prescalar bits in the control registers of the particular timer counter implemented.
			
			This function returns \textbf{\texttt{0}} always.
			
\subsection{\textbf{\texttt{enable\_tov\_interrupt}} Function}
			\label{ss:HALtcenable_tov_interruptFunction}
			
			The \textbf{\texttt{enable\_tov\_interrupt}} function facilitates the creation of function execution upon the timer counter overflowing. This is accomplished by providing the function with a pointer to a user-created interrupt service routine (ISR) that will be called at the appropriate time.
			
			The function first clears all waveform generation (WGM) bits in the timer counter control register and then enables the Timer Overflow Interupt Enable (TOIE) bit in the timer interrupt mask register. This allows the counter to increment from \textbf{\texttt{0x00}} to \textbf{\texttt{0xFF}} (or \textbf{\texttt{0xFFFF}} in the case of 16-bit timers); with the overflow flag being initiated upon the counter resetting to \textbf{\texttt{0x00}}.
			
			The function pointer to the user's ISR is placed in the appropriate element of a private global function array. This allows the function to be called when the ISR is fired.
			
			This function returns \textbf{\texttt{0}} always.
			
\subsection{\textbf{\texttt{disable\_tov\_interrupt}} Function}
			\label{ss:HALtcdisable_tov_interruptFunction}
			
 			Disabling the timer overflow interrupts on a particular timer is catered for by this function. 
 			
 			The function performs this task by clearing the Timer Overflow Interrupt Enable bit in the timer counter interrupt mask register and setting the appropriate element of the interrupt function array with NULL. 
			
			This function returns \textbf{\texttt{0}} always.
			
\subsection{\textbf{\texttt{enable\_oc}} Function}
			\label{ss:HALtcenable_ocFunction}
			
 			The \textbf{\texttt{enable\_oc}} function allows the user to initiate output compare operation on a timer implementation. The choice of modes are governed by the \textbf{\texttt{tc\_oc\_mode}} enumerated list that is found in the \textbf{\texttt{tc.h}} file. These modes pertain directly to those listed in the ATmega2560's datasheet.
 			Note that the \textbf{\texttt{enable\_oc}} function calls one of two secondary functions depending on whether the timer implemented is a 8 or 16-bit type. The two different functions called are \textbf{\texttt{enable\_oc\_8bit}} and \textbf{\texttt{enable\_oc\_16bit}} respectively and each is required to take into account the variety of different modes of operation that are possible on the 8 and 16-bit timers.
			
			This function returns \textbf{\texttt{0}} always.

\subsection{\textbf{\texttt{enable\_oc\_channel}} Function}
			\label{ss:HALtcenable_oc_channelFunction}
			
			The \textbf{\texttt{enable\_oc\_channel}} function controls the initialisation and settings relating to a channel within a timer. This function controls the mode of the output compare pins associated with each timer channel, attempts to procure the required GPIO semaphores and sets the appropriate direction for these pins.
			
			Two input values are provided to the function. These are the desired channel to enable, and the mode to apply to it. Both of these variables are found in the enumerated lists \textbf{\texttt{tc\_oc\_channel}} and \textbf{\texttt{tc\_oc\_channel\_mode}} which are found in the \textbf{\texttt{tc.h}} file.
			
			The stanadard format of the \textbf{\texttt{TCCRnA}} register for AVR timers allows a simple arithmetic process to be used for accessing the compare output mode bits that set the function of the the ouput pins. This includes a \textbf{\texttt{COM\_BIT\_OFFSET}} that represents the number of bits associated with each pin. This is then multiplied by the channel number provided as an input value to the function to allow access to the correct bits for each channel.
			
			A return value of \textbf{\texttt{0}} indicates successful completion of the function, whilst a value of \textbf{\texttt{-1}} indicates that the GPIO semaphores were unable to be procurred or an invalid mode was specified.
			
\subsection{\textbf{\texttt{enable\_oc\_interrupt}} Function}
			\label{ss:HALtcenable_oc_interruptFunction}
			
			Attaching or enabling an ISR function to respond to an output compare interrupt is accomplished with this function.
			
			The function expects two input values, being the timer channel to which the interrupt is being attached and a function pointer to the user-defined ISR.
			
			Internally, the function places the pointer in the appropriate element of the interrupt function pointer array and enables the Output Compare Interrupt Enable (\textbf{\texttt{OCIE}}) bit in the timer interrupt mask register.
			
			This function returns \textbf{\texttt{0}} if the ISR was successfully attached and \textbf{\texttt{-1}} if an invalid channel was applied to the implemented timer.
			
\subsection{\textbf{\texttt{disable\_oc\_interrupt}} Function}
			\label{ss:HALtcdisable_oc_interruptFunction}
			
			This function is used for de-associating ISR routines with the associated interrupt.
			
			Only one value is expected for the function and that is the channel of the implemented timer to disable.
			
			The interrupt routine is disabled by replacing the function pointer in the interrupt function array with NULL and clearing the appropriate \textbf{\texttt{OCIE}} bit in the timer interrupt mask register.
			
			This function returns \textbf{\texttt{0}} if the ISR was successfully disabled and \textbf{\texttt{-1}} if an invalid channel was applied to the implemented timer.
			
\subsection{\textbf{\texttt{set\_icR}} Function}
			\label{ss:HALtcset_icRFunction}
			
			This function allows the input capture register of suitably equipped timers to be manually set for output compare modes where this register is utilised as the timer's \textbf{\texttt{TOP}} value.
			The function places the provided value in the \textbf{\texttt{ICR}} register and returns 0 always.
			
\subsection{\textbf{\texttt{enable\_ic}} Function}
			\label{ss:HALtcenable_icFunction}
			
			This function enables the input capture functionality of a timer when it is provided a channel and mode value. These values represent one of the options defined in the \textbf{\texttt{tc\_ic\_channel}} and \textbf{\texttt{tc\_ic\_mode}} enumerated lists within \textbf{\texttt{tc.h}}.
			
			The function enables this functionality by attempting to procure the required GPIO semaphores and setting or clearing the the input capture enable and noise-cancelling in the timer control registers as the specified mode dictates.
			
			If the GPIO semaphores are successfully procured, the GPIO pin is set as an input and a value of \textbf{\texttt{0}} is returned. Otherwise, a value of \textbf{\texttt{-1}} is returned for an invalid mode or un-available semaphore.
			
\subsection{\textbf{\texttt{enable\_ic\_interrupt}} Function}
			\label{ss:HALtcenable_ic_interruptFunction}
			
			The \textbf{\texttt{enable\_ic\_interrupt}} function allows the user to register a ISR routine to an input capture channel associated with the implemented timer (when one is available).
			
			The input capture channel is provided by way of a channel number (defined in the \textbf{\texttt{tc\_ic\_channel}} enumerated list) as an input value, along with a function pointer to the user's ISR.			
			Internally, the function places the function pointer provided in the appropriate element of the global function pointer array and sets the Input Capture Interrupt Enable \textbf{\texttt{ICIE}} bit in the timer interrupt mask register.
			
			The functionre returns \textbf{\texttt{0}} for success and \textbf{\texttt{-1}} to indicate that input capture functionality is not available on the implemented timer.	
			
\subsection{\textbf{\texttt{disable\_ic\_interrupt}} Function}
			\label{ss:HALtcdisable_ic_interruptFunction}
			
			In order to disable an interrupt that is attached to an input capture event, this function may be used. It once again accepts the channel for which the interrupt should be disabled as an input value.
			
			When called, the function replaces the ISR pointer in the global interrupt function array with NULL and clearies the \textbf{\texttt{ICIE}} in the timer interrupt mask register. The semaphore for the input capture pin is also vacated before a value of \textbf{\texttt{0}} is returned. If the implemented timer does not feature input capture functionality, this function will return a value of \textbf{\texttt{-1}}.
			
\subsection{\textbf{\texttt{get\_icR}} Function}
			\label{ss:HALtcget_icR_interruptFunction}
			
			When provided with a valid input capture channel number and an appropriate data type template is provided in the function call, this function will return the contents of the \textbf{\texttt{ICR}} register.	
			
\subsection{\textbf{\texttt{vacate}} Function}
			\label{ss:HALtcvacateFunction}
			
			This function may be called when the timer implementation is to be reset, or else it is called by default when the implementation goes out of scope.
			
			The function stops the timer by clearing the prescalar and prevents any ISR functions from firing by clearing the interrupt mask register and resetting the ISR function pointers to NULL.
			
			The semaphores acquired for the relevant GPIO pins are vacated iteratively and the implementation pointer is also reset.
			
\subsection{ISR Pointers}
			\label{ss:HALtcISRPointers}
			
			These are described briefly for future maintenance purposes. Each of the available interrupt types for the Timer/Counters has an element within the private global \textbf{\texttt{timerInterrupts}} array. These elements are addressed via an enumerated list of the same interrupt types, \textbf{\texttt{timer\_interrupts}}, which is contained in \textbf{\texttt{tc.cpp}}.
			
			When a user supplied pointer to an application based ISR is placed in the function array, it may then be called via the AVR defined \textbf{\texttt{ISR}} macro when the enabled interrupt is fired. This macro is defined in \textbf{\texttt{avr/interrupts.h}} along with the platform-specific interrupt vectors.
			
			The macros are employed along with all the available interrupt vectors at the end of \textbf{\texttt{tc.cpp}}. Note that at least on of these interrupt vectors is selectively compiled depending on whether the configuration is set to Bare Metal or FreeRTOS. This is due to multiple instances involving the same interrupts vector causing compile errors.
			
\section{\textbf{\texttt{watchdog}} Module}
			\label{s:HALwatchdogModule}
			
			The \textbf{\texttt{watchdog}} module provides abstracted use of the \textbf{\texttt{avr/wdt.h}} functionality on AVR targets. It allows a user to enable, pat and disable a watchdog timer easily.

\subsection{\textbf{\texttt{enable}} Function}
			\label{ss:HALwatchdogenableFunction}
			
			To enable a watchdog timer, the user simply calls this function and provides it one of a number of pre-defined time out periods. These periods are defined in \textbf{\texttt{watchdog.h}}.
			
			The function enables a watchdog timer with a call to the AVR function {\textbf{\texttt{wdt\_enable}} and returns {\textbf{\texttt{0}} always.
			
\subsection{\textbf{\texttt{pat}} Function}
			\label{ss:HALwatchdogpatFunction}
			
			In order to prevent a watchdog initiated reset, the {\textbf{\texttt{pat}}} function calls the AVR function {\textbf{\texttt{wdt\_reset}}}.
			
\subsection{\textbf{\texttt{disable}} Function}
			\label{ss:HALwatchdogdisableFunction}
			
			This function simply disables the watchdog timer in use by calling the \textbf{\texttt{wdt\_disable}} AVR function.
			
\section{\textbf{\texttt{usart}} Module}
			\label{s:HALusartModule}
			
			This module provides functionality for implementing, initialising and using the USART hardware onboard AVR platforms.
			
\subsection{\textbf{\texttt{grab}} Function}
			\label{ss:HALusartgrabFunction}
			
			This function may be used for creating a \textbf{\texttt{usart}} isntance with which to initialise and use. Only one variable is expected as input to these function, namely a \textbf{\texttt{usartNumber}} value which is of the enumerated list \textbf{\texttt{usart\_number}} and found in \textbf{\texttt{usart.h}}.
			
			If called for the first time the function initialises the USART implementations by applying the relevant register locations in memory to the implementations contained in the \textbf{\texttt{usart\_imps}} array. The initialisation process also applies the GPIO addresses of the target's USART TX/RX pins and the addresses of the semaphores associated with these pins. The values for these memory locations and addresses may be found at at the start of the \textbf{\texttt{usart.cpp}} file.
			
			The \textbf{\texttt{grab}} function is then responsible for procuring the semaphores for the USART module's TX/RX GPIO pins. If this process is successful, a link to the requested implementation is created and a \textbf{\texttt{usart}} instance returned, otherwise \textbf{\texttt{NULL}} is returned.
			
\subsection{\textbf{\texttt{set\_mode}} Function}
			\label{ss:HALusartset_modeFunction}	
			
			The USART modules are capable of operating in multiple modes. This function allows the operation to be set and the options possible are listed in the \textbf{\texttt{usart\_setup\_mode}} list contained in \textbf{\texttt{usart.h}}.
			
			When called, this function disables global interrupts to allow for un-disturbed access to the control registers. These registers are then accessed and the settings applied as dictated by the specified mode.
			
			For modes such as \textbf{\texttt{MASTER\_SPI}} that require a GPIO pin whose directionality is not automatically set through hardware, this function is also responsible for this process.
			
			Finally, the transmission and reception circuitry within the USART hardware is activated for use.
			
\subsection{\textbf{\texttt{set\_frame}} Function}
			\label{ss:HALusartset_frameFunction}
			
			The ability to define the serial format required for communication is extremely important. This function caters for this by allowing the user to specify their desired settings for packet size, parity type and number of stop bits. The parity type should be one of the options listed in the \textbf{\texttt{parity\_option}} enumerated list within \textbf{\texttt{usart.h}}. The frame size should be between five and eight bits whilst the user should specify either one or two stop bits.
			
			The function applies the desired packet settings by manipulating the appropriate USART control registers. These settings apply to the ATmega2560 and should be applicable throughout most 8-bit platforms but may require slight editing.
			
			The function returns \textbf{\texttt{0}} if the function completes the task successfully, otherwise \textbf{\texttt{-1}} is returned to indicate an invalid option was specified.
			
\subsection{\textbf{\texttt{set\_baud\_rate}} Function}
			\label{ss:HALusartset_baud_rateFunction}
			
			The baud rate is another important initialisation setting for the serial communication. The user may specify whatever baud rate they wish.
			
			This value is then used to evaluate the the required value to be placed in the USART Baud Rate Register (\textbf{\texttt{UBRR}}). The configured mode of the USART hardware also has an effect on this value, as does the micropcrocessor's CPU frequency. 
			
			This function returns \textbf{\texttt{0}} always.

\subsection{\textbf{\texttt{double\_usart\_speed}} Function}
			\label{ss:HALusartdouble_usart_speedFunction}
			
			Doubling the USART speed is an option that is only pertinent when the USART module has been configured in asynchronous operation. This function accepts a boolean value to allow the user to either enable or disable such operation depending on the value being true or false respectively.
			
			The function clears the Data Register Empty \textbf{\texttt{UDRE}} flag whilst accessing the control registers per the isntructions in the ATmega2560 datasheet.
			
			The function returns \textbf{\texttt{0}} always.

\subsection{\textbf{\texttt{buffer\_is\_available}} Function}
			\label{ss:HALusartbuffer_is_availableFunction}
			
			This function returns a boolean value to indicate whether the USART data register buffer is empty or not at the time of checking. A true value indicates that the buffer is available, whilst false indicates otherwise.
			
\subsection{\textbf{\texttt{transmit\_byte}} Function}
			\label{ss:HALusarttransmit_byteFunction}
			
			This function allows the user to transmit a single byte value via the established USART connection by copying it to the data register buffer. This function also internally checks to ensure the buffer is free before copying the value as undefined behaviour can occur if data is copied into the buffer before it is free.
			
			The function returns \textbf{\texttt{0}} always.
			
\subsection{\textbf{\texttt{transmit\_string}} Function}
			\label{ss:HALusarttransmit_stringFunction}
			
			This function has the capability to transmit a NULL-terminated string (or array) via USART. The user should provide a pointer to this string which is then transmitted in a byte-by-byte fashion. Note that this function will 'block' the processor until the transmission is complete.
			
			The function returns \textbf{\texttt{0}} always.
			
\subsection{\textbf{\texttt{transmit\_array}} Function}
			\label{ss:HALusarttransmit_arrayFunction}
			
			To transfer an array of arbitary size and type, this function may be employed. It has the ability to transmit such data in a byte-by-byte fashion until the routine is complete.
			
			A pointer cast to the array of data that has been cast to a fixed-width byte datatype is required, as is the number of elements to transmit.
			
			This function simply makes successive calls to the \textbf{\texttt{transmit\_byte}} function and blocks all CPU resourses until it has completed. A value of \textbf{\texttt{0}} is always returned. 
			
\subsection{\textbf{\texttt{transmission\_complete}} Function}
			\label{ss:HALusarttransmission_completeFunction}
			
			This function returns a boolean to indicate whether there is new data that has not been transmitted in the buffer of the USART channel implemented is empty or not. The value returned is true if there is no new data and false otherwise.
			
\subsection{\textbf{\texttt{receive\_complete}} Function}
			\label{ss:HALusartreceive_completeFunction}
			
			This function returns a boolean to indicate whether the data buffer of the USART channel implemented contains new unread data. The value returned is true if this is so and false otherwise.
			
\subsection{\textbf{\texttt{receive\_byte}} Function}
			\label{ss:HALusartreceive_byteFunction}
			
			In order to receive data via USART, the data register buffer must be read when new unread data is present. This function checks to ensure there is in face unread data and returns the contents of the buffer if there is.
			
\subsection{\textbf{\texttt{enable\_interrupt}} Function}
			\label{ss:HALusartenable_interruptFunction}	
			
			A user defined ISR may be attached to a number of interrupt vectors associated with the USART hardware. These interrupt vectors or types are contained in the \textbf{\texttt{usart\_interrupt\_types}} enumerated list of the \textbf{\texttt{usart.h}} file.	Broadly speaking, these interrupts can occur when a USART transmission has been completed, when new data has been received, and when the data register buffer is empty.
			
			When called, the function places the provided function pointer in the appropriate element of the function pointer array and enables the relevant interrupt enable bit in the USART control register.
			
\subsection{\textbf{\texttt{disable\_byte}} Function}
			\label{ss:HALusartdisable_byteFunction}
			
			Any interrupt functions associated with a USART interrupt signal can be disabled by calling this function.
			
			The user must provide the one of the enumerated interrupt types available for the USART to disable a specific interrupt. This value is used to set the appropriate element of the interrupt function array to NULL.
			
			Lastly, the appropriate interrupt enable bit is cleared in the control registers belonging to the USART.
			
\subsection{\textbf{\texttt{usart\_error}} Function}
			\label{ss:HALusart_errorFunction}
			
			Communication via USART may produce cases where the communicated packet is invalid or corrupted. This function provides the ability to check any errors on the USART connection.
			
			The function checks the status bits in the USART control register and iteratively masks the bits relating to frame errors, data over run errors and parity errors. If any of these errors exist, it will return an indicating flag, as listed in the \textbf{\texttt{usart\_error\_types}} list located in \textbf{\texttt{usart.h}}, for the first error found; or else a flag indicating no error if none exists.
			
\subsection{\textbf{\texttt{vacate}} Function}
			\label{ss:HALusartvacateFunction}
			
			The \textbf{\texttt{vacate}} function may be called when the USART instance is no longer required; or else when it goes out of scope.
			
			This function vacates the semaphores that were procured upon initialisation for the required GPIO pins and sets all the interrupt function pointers associated with the instance to NULL.
			
			The USART transmission and reception circuitry is deactivated before the pointer to the implementation is severed by resetting it to NULL.

\subsection{Master SPI Mode Functions}
			\label{ss:HALusartMasterSPIModeFunctions}
			
			The USART hardware on the ATmega2560 device has the ability to operate in a SPI manner, emulating the protocol of this communication bus. This has some advantages over the conventional SPI mode including the double buffered transmitter (the SPI transmitter has no buffer).
			
			Access to the Master SPI Mode (MSPIM) is via the \textbf{\texttt{set\_mode}} function (see: \S\ref{ss:HALusartset_modeFunction}).
			
\subsubsection{\textbf{\texttt{set\_spi\_mode}} Function}
			\label{sss:HALusartMasterSPIModeset_spi_modeFunctions}
			
			This function allows the user to set the desired clock phase and polarity settings for MSPIM. The options are contained in the enumerated list \textbf{\texttt{mspim\_mode}} (found in \textbf{\texttt{usart.h}} and relate directly to those modes specified for MSPIM in the ATmega2560 datasheet.
			
			These modes are applied only if the USART hardware has been previously initialised to MSPIM. If this is not the case, a value of \textbf{\texttt{-1}} is returned. 
			
			Depending on the mode selected, the function will manipulate set or clear the clock polarity and phase select bits in the USART control registers.
			
\subsubsection{\textbf{\texttt{set\_bit\_order}} Function}
			\label{sss:HALusartMasterSPIModeset_bit_orderFunctions}
			
			This function allows the user to set the desired bit order of the MSPIM connection. The options are one of \textbf{\texttt{MSPIM\_MSB\_FIRST}}, or \textbf{\texttt{MSPIM\_LSB\_FIRST}} (defined in \textbf{\texttt{usart.h}}), which relate to either setting the most or the least significant bit first. Note, this function will only work if the USART hardware has been previously initialised to MSPIM. If this is not the case, a value of \textbf{\texttt{-1}} is returned. 
			
			The bit order is set by manipulating the USART control registers to set or clear the USART Data Order (\textbf{\texttt{UDORD}}) bit.
			
			A value of \textbf{\texttt{0}} is returned if the function is successful.
			
\subsubsection{\textbf{\texttt{spi\_transfer\_byte}} Function}
			\label{sss:HALusartMasterSPIModespi_transfer_byteFunctions}
			
			When in MSPIM mode, the transfer of a single byte can be achieved by this function. A single byte is provided as an input argument and the function returns the byte it receives in the transfer process, or else a value of \textbf{\texttt{0}} if the USART hardware has not been configured in MSPIM.
			
			Internally, the function waits for the data register to become free by checking the \textbf{\texttt{UDRE}} flag. When the register is free, the outgoing data is passed into the buffer. The function then waits for the incoming data to be received, and returns this byte when it is available.
			
			Note that this function blocks the resources of the microprocessor until the function completes.
			
\subsubsection{\textbf{\texttt{spi\_transfer\_array}} Function}
			\label{sss:HALusartMasterSPIModespi_transfer_arrayFunctions}
			
			An array of arbitary size and type can be received and transmitted in MSPIM via this function. A pointer to both the array to be transmitted and a location where the received data may be returned is required as well as the number of elements to transfer.
			
			The function sequentially transfers each of the byte elements of the array via the \texttt{spi\_transfer\_byte}} function (see: \S \ref{sss:HALusartMasterSPIModespi_transfer_byteFunctions}) and places in received data at the memory address of the pointer provided to it.
			
			Note that this function blocks the resources of the microprocessor until the function completes.
			
\subsubsection{\textbf{\texttt{spi\_dma\_transfer\_array}} Function}
			\label{sss:HALusartMasterSPIModespi_dma_transfer_arrayFunctions}
			
			This function has not been implemented on the ATmega2560 platform due to the lack of a DMA or similar device. However, when implemented it is envisaged to behave very similarly to the \textbf{\texttt{spi\_transfer\_array}} function whilst facilitating a non-blocking type function.
			
			This will potentially be of use in datalogging applications and/or interfacing with external flash memory.

\section{\textbf{\texttt{spi}} Module}
			\label{s:HALspiModule}
			
			The Serial Parallel Interface (SPI) module currently consists of an interface and conceptual paltform to build the implementation on.
			
			The interface has been envisaged to provide appropriate and useful functionality for all required devices that support this functionality. This includes creating, initialising and using SPI connections on the device.
			
			Creating the SPI connection is performed by 'grabbing' an SPI implementation with the \textbf{\texttt{grab}} function. This function should perform the bare minimum initialisations that are necessary such as assigning register addresses in memory and registering the GPIO addresses of the pins required for the SPI bus.
			
			Initialisation processes should allow for the setting of crucial parameters to do with the SPI communication bus. The interface currently contains a number of functions for this purpose, including \textbf{\texttt{set\_data\_mode}}, \textbf{\texttt{set\_bit\_order}} and {\texttt{set\_clock\_divider}}.
			
			These functions should be implemented by accessing and manipulating the SPI control registers on the AVR platform. Use of macros such as \textbf{\texttt{\_SFR\_IO8}} and \textbf{\texttt{\_SFR\_MEM8}} in conjunction with the registers memory address should allow more direct access and less reliance on the built in AVR variables that contribute to larger compiled file sizes.
			
			The transferral of data over the SPI bus is envisaged to take two different forms; blocking and non-blocking processes. The blocking operation will consume the processor's resources until the transferral has been completed whilst the alternate option will most likely utilise a form of Direct Memory Access (DMA) controller.
			
			The interface includes prototypes for blocking functions that are capable of tranferring a single byte of data or an array of elements cast to a fixed-width byte data type (\textbf{\texttt{transfer}} and \textbf{\texttt{transfer\_array}} respectively). 
			
			There is also provision for transferring an array of arbitary size and type via the DMA controller with \textbf{\texttt{dma\_transfer}}. This function will require implementation of a DMA process and therefore only be applicable on the platforms that can support one, for example the Atmel AT32UC3C0512C.
			
			Functionality has been created to cater for dealing with multiple SPI slaves on one or more buses. This is carried out by the user accessing the \textbf{\texttt{spi\_slave}} class. Using this class allows for multiple slaves to be 'attached' at a specific GPIO address and then enabled or disabled individually for communication. This allows multiple slaves to be 'daisy-chained' on a single SPI bus that might only provide one Slave Select pin.
			
\section{\textbf{\texttt{i2c}} Module}
			\label{s:HALi2cModule}
			
			 The \textbf{\texttt{i2c}} module interface has been created to cater for all the functionality required of the Inter-Integrated Circuit (I$^{2}$C) communication protocol.
			 
			 As it stands, the interface contains the ability for the user to initialise an instance of the \textbf{\texttt{i2c}} class and configure an \textbf{\texttt{i2c}} bus as the master device. This configuration may be carried out by the \textbf{\texttt{set\_bit\_rate}} function.
			 
			 The control of starting and stopping communication on the I$^{2}$C bus as well as checking for acknowledgement from the slave devices shoudl be abstracted away from the user as the current interface reflects. Rather, the implementation of the \textbf{\texttt{i2c}} module should manage these tasks automatically when the \textbf{\texttt{transmit}} and \textbf{\texttt{receive}} functions are called.
			 
			 The \textbf{\texttt{transmit}} and \textbf{\texttt{receive}} functions in the interface currently feature the ability to allow the user to optionally address specific locations on the slave devices they are communicating with. If implemented, this will give the module the ability to communicate with a wider range of peripherals.
			
\section{Size and Use}
			\label{s:HALSizeAndUse}
			At this stage, the HAL (for the AVR implementation) compiles to approximately 17.3kB, making it too large to fit into the bootloader section. Currently the smallest flash size of any of the supported microcontrollers is 64kB (the ATmega64M1), meaning it can be used for any of the specified ValleyForge microprocessors. However, safety mechanisms in the code, as well as multiple layers of abstraction mean that it does not run efficiently.	
						
\chapter{Bootloader}
			\label{C:Bootloader}
			The bootloader is used to transfer code to the microprocessor without use of an ISP programmer. Currently an AVR bootloader has been started, and its functionality is to be able to run a bootloader program, then switch to the application code when a button is pushed.
			There is a useful guide on writing bootloaders at \url{http://www.avrfreaks.net/index.php?name=PNphpBB2&file=viewtopic&t=79206}. You need to be logged in to access it though.			 
\section{General Design}
			\label{s:BootloaderGeneralDesign}
			So far only the AVR bootloader has been implemented. It has one main \texttt{bootloader.cpp} file, which takes care of writing to flash, flashing LEDs, running timers, receiving a signal of new available firmware, and switching to the application. Used in conjunction with this is the bootloader module. There are a few different modules corresponding to different transfer protocols. Which module is used depends on which is specified in the component's configuration file. 
			The bootloader calls module functions for the transfer of data from the remote device. It then writes the flash a page at a time. When it is finished, it sets the microprocessor's register back to a default state and runs the application.
			
\section{Boot Address}
			\label{s:BootloaderBootAddress}
The boot address given in the datasheets is a \textbf{word} address, not a \textbf{byte} address. However the address entered into the makefile via the \texttt{build\_configs.cfg} file is a \textbf{byte} address, and therefore the address needs to be converted. Take this into account if you want to change the boot size and therefore need a new boot address, or a re-implementing the bootloader. Make sure all the fuses are correct. Things that can happen with an incorrect boot address:

\begin{itemize}
\item The bootloader program may not run at all and go straight to the application code. This usually happens when the boot address given is lower than the actual. It means the program counter No-Ops all the way round to the starting address again, without ever running the bootloader code.
\item Interrupts may not work. When you write a bootloader, it creates a new interrupt table at the bootloader start address. This means you will have two interrupt vector tables in your microprocessor. An instance occurred where the bootloader address was set too high. The bootloader code still ran, but because the interrupt vector table was not where the microprocessor expected it to be, the interrupts did not fire.
\end{itemize}

\section{Fuses}
			\label{s:BootloaderFuses}
			Make sure that the microprocessor's fuses are set to the correct values. This can not be done through a bootloader, it must be done using an ISP, JTAG, or HVPP programmer. The BOOTRST fuse is of course, important, but one must also make sure that the boot size is correct.
			
			Be careful, also, to make sure the fuses aren't set to some strange clock speed. This bricks the micro. It can be fixed using HVPP, but is a right pain.
						
\section{Registers}
			\label{s:BootloaderRegisters}
			When the bootloader program has finished, make sure that before it starts running the application code, it resets hardware registers back to their original state. This can be done manually, or by resetting the micro (using the watchdog timer) and then going straight into the application code.
			
			
\section{Merging Hex Files}
			\label{s:BootloaderMergingHexFiles}
			The application and bootloader are built seperately, so to merge them into one hex file, one simply deletes the last line of one hex file and puts the other after it in a text editor.
			There are also scripts online that will do it for you.
			
\chapter{FreeRTOS}
			\label{C:FreeRTOS}
			FreeRTOS is a small operating system for microcontrollers, that has been ported to work with all the Atmel processors being used for the ValleyForge project. Its website can be found at \url{http://www.freertos.org/}
			
\section{Structure}
			\label{s:FreeRTOSStructure}
			FreeRTOS is split into two main sets of files: Common and Port-Specific. The common files contain all port-independent elements, those describing tasks and task handling.\newline
			There are three source files that form the FreeRTOS base: tasks.c, queue.c, and list.c. On top of this, there are a bunch of common header files that are included by these source files.\newline
			There are four port-specific files for each port. The port source file, port.c, a memory allocation file (this is not directly port-specific, there are three different types, and one uses the one best suited to one's microcontroller), and two header files, portmacro.h and projdefs.h. 			
						
			
\section{Current Status}
			\label{s:FreeRTOSCurrentStatus}
			FreeRTOS has been made to run on all Atmel devices covered in the ValleyForge toolchain. Each was tested by running a single task that flashed an LED. There are many different options and variations within FreeRTOS, and not all have been tested or implemented. \newline
			Some playing around with clock settings and such is required. This should be editable in FreeRTOSConfig.h, which is placed in the user's source directory when they create a component that uses FreeRTOS.\newline
			The ATmega2560 port is different from the other AVR ports, as it needs to address a larger flash space. This has some strange effect. In any case, I had to download the port from AVR freaks, which used an old version. I finally managed to get the port working with the new version as it does now, but if the user tries to implement any other FreeRTOS functionality on the ATmega2560 and it doesn't work it may have something to do with this. Also, the person that ported this didn't include support for Co-Routines, so if you want to implement them you may have to do some tinkering.
			
			
\chapter{Programming with the STK600}
			\label{C:STK600}
			For debugging purposes, it may be that you will be using the STK600. I have decided to try and impart the wisdom I have learnt about this fickle device here.
			
\section{Clock and Voltage}
			\label{STK600ClockVoltage}
			The clock speed and Target Voltage of the STK600 can be set using avrdude, or AVR Studio. Most devices operate between 3 and 5.5V, which can be supplied through the USB.\newline
			If the status LED is blinking orange, this means that you may have the wrong combination of routing and socket boards mounted. If it is blinking red, officially this means it is drawing too much current and that you have a short circuit on your board, but actually it just seems to do this at its whim, and still works, so I wouldn't read too much into it.			
			
\section{Programming AVR Devices}
			\label{s:STK600AVRDevices}
			To upload a hex file using avrdude with the STK600, I:\newline
			
			Attached the 6-pin header for ISP programming, then type into the terminal:
			
\begin{lstlisting}[frame=trBL, breaklines=true]
 avrdude -c stk600 -p m64 -P usb -F -v -U myfile.hex 
\end{lstlisting}

			The ``-F'' flag near the middle is the ``Force'' flag. This was because there was no mcu code for the ATmega64M1, so I used that given for the ATmega64, and used the force flag to program the device despite its warning that it didn't have the right identification code. \newline
			As stated in the bootloader description (section \S\ref{C:Bootloader}), make sure that if you are loading a bootloader on, that the linker has been given the correct bootloader start address.\newline
			Sometimes, having certain ports attached to LEDs or switches can cause errors when programming the micros using the ISP programmer.
			
			
\section{Programming AVR32 Devices}
			\label{s:STK600AVR32Devices}
			Avrdude does not provide any support for AVR32 devices. For this reason I used avr32program. This used to come with the AVR32 Toolchain, but for some reason no longer does. It does come with the latest version of AVR32 Studio for Linux however, which is where I got it from.

			This is how I loaded code:\newline
			
			Attach the JTAG connection using the 10-pin header.
 						
\begin{lstlisting}[frame=trBL, breaklines=true]
avr32program -c USB:0048395C7318 --part UC3C0512C program -finternal@0x80000000,512Kb -v -e -F elf myfile.elf
\end{lstlisting}			
			 
			 While avrdude automatically erases the flash. before writing unless told not to, avr32program does not erase unless told to. Therefore we need the ``-e'' flag to tell it to erase the the flash.\newline
			 This is where I had the most trouble with having things connected to ports when the chip was starting up/being written to. Port B in particular was not very favourable.


\section{Other Observations}
			\label{s:STK600OtherObservations}			 
			 Occaisionally the device will throw a fit and randomly flash LEDs that it supposed to be giving a solid signal to. Resetting the STK600 fixes this. \newline
			 Generally ignore the status LED.\newline
			 Don't ignore the VTarget LED. This indicates whether there is voltage to the device.\newline
			 Be careful, make sure you know what you're doing before changing fuse values.\newline
			 Make sure you take anti-static precautions when changing routing/socket boards.
			 
\chapter{FAQ}
			\label{C:FAQ}
			Of course we don't really have any frequently asked questions at this stage, but I'll just post up a few things here that I might expect. I'm guessing throughout 2012 I may have to update this section, depending on how much people will be asking questions, but anyway:
			
\section{I want to add a new HAL module. How do I do this?}
			\label{s:FAQNewHAL}
			The first thing you'll want to do is decide what functions are going to be available to the user. What will they require? Is it possible to implement these things on all the microprocessors? If not, which ones will it work on\newline
			Next, create your header file. A good example of the type of header file one could create would be \texttt{gpio.h}, found in \texttt{res/common/hal}. You do not need to have an instance of the thing you are trying to manipulate, particularly if it is just one standalone peripheral. Static functions would be fine, or find another way of doing it.\newline
			Next you'll want to implement it for whichever microprocessor you happen to be using at that time. If the module will never be implemented for some microprocessors, use the avr-gcc preprocessor to \texttt{\#ifdef} the code to produce an error if a user tries to compile it for the wrong processor.\newline
			Now add the module to the list in the relevant build configuration, in the field ``HAL\_EN\_LIST''. When you compile with that build configuration, it should now compile your new module. You will be able to debug your module from there.
			
\section{I want to add a new input parameter to a script. How do I do this?}
			\label{s:FAQNewParams}
			This is described in section \S\ref{s:ScriptParameters}
			
\section{How do I add a new build parameter used in a makefile?}
			\label{s:FAQNewMakeParams}
			First, you will need to edit the makefile to include what you want. This may be something like
			
\begin{lstlisting}[frame=trBL, breaklines=true]
MYNEWPARAMETER=<<<BUILD_INSERTS_MYNEWPARAMETER_HERE>>>
\end{lstlisting}			

			and obviously this is going to be used somewhere later in the makefile.\newline
			
			Next, you will want to enter the value into the relevant build configurations. Open the \texttt{build\_configs.cfg} script (in \texttt{bld/build}), and add MYNEWPARAMETER=Seventytwelvemillion to the build configurations that you want this parameter for (you can have it in the others as an empty parameter if you wish).
			You should also make sure that the value of the parameter is clear by the script \texttt{clear\_bconf\_vars}. Open the script, and add the name of the parameter to the list.\newline
			
			Navigate to the \texttt{build} script, and to the function that uses the makefile you are interested in (for example, \texttt{makeavr}).
			Enter the line amongst the other similar lines:
\begin{lstlisting}[frame=trBL, breaklines=true, language=bash]			
sed -i "s^\(CC *= *\)BUILD_INSERTS_MYNEWPARAMETER_HERE^\1$MYNEWPARAMETER^" $TCPATH/$TMP_SRC_DIR/${COMPONENT}/Makefile 
\end{lstlisting}				

This will insert your parameter into the makefile at build time.

\section{There is an error during building which says something like ``Undefined reference to ``<<<TC\_INSERTS\_NAME\_HERE>>>>''''.}
			\label{s:FAQUndefinedError}
			This is a problem with the preprocessing substitution. Search the preprocessing functions for the reference, and find out why it is not replacing it. It could be that you haven't specified it to be replaced with anything, it could be a difference in spelling, it could be the wrong function.
			
\section{FreeRTOS Won't Compile}
			\label{s:FAQFreeRTOS}
			Make sure all the common files are up to date. Also, make sure that the Application and Idle hooks are present in the component file. While these are configured to be active in the \texttt{FreeRTOSConfig.h}, they are required to be present in the component.
			 
			 

			
	% THEN COMES ANY SUPPORTING MATERIAL.

	% Switch to 'appendices' mode.  Alphabetic chapter headings.
	\appendix



\end{document}
